\documentclass[UTF8,12pt,a4paper]{ctexart}
\usepackage[utf8]{inputenc}
\usepackage{color}
\usepackage{array}
\usepackage{diagbox}
\usepackage{multicol}
\usepackage{multirow}
\usepackage{CJKfntef}
\usepackage{booktabs}
\usepackage{fancyhdr}
\usepackage{graphicx}
\usepackage{lastpage}
\usepackage{indentfirst}
\usepackage{amsmath,amssymb}
\usepackage{tabu}
\usepackage[colorlinks,linkcolor=blue,anchorcolor=blue,citecolor=green,bookmarks=true]{hyperref}
%\usepackage[breaklinks,colorlinks,linkcolor=black,citecolor=black,urlcolor=black]{hyperref}
\usepackage{bookmark}
\usepackage{inconsolata}
\usepackage{geometry}
\usepackage{titlesec}
\usepackage{paralist}
\usepackage[ruled,linesnumbered]{Algorithm2e}
\setlength{\parindent}{2em}
\geometry{left=2.45cm,right=2.45cm,top=2.75cm,bottom=2.6cm}
\setcounter{secnumdepth}{0}
\let\itemize\compactitem
\let\enditemize\endcompactitem
\let\enumerate\compactenum
\let\endenumerate\endcompactenum
\let\description\compactdesc
\let\enddescription\endcompactdesc

\usepackage{booktabs} % for much better looking tables
\usepackage{array} % for better arrays (eg matrices) in maths
\usepackage{paralist} % very flexible & customisable lists (eg. enumerate/itemize, etc.)
\usepackage{verbatim} % adds environment for commenting out blocks of text & for better verbatim
\usepackage{subfig} % make it possible to include more than one captioned figure/table in a single float
\usepackage{xcolor}
\usepackage{endnotes}
\usepackage{multirow}
\usepackage{notoccite}
\usepackage{longtable}
\usepackage{geometry}
\usepackage{multicol}
\usepackage{multirow}
\usepackage{tabu}
\usepackage{xeCJK}
\usepackage{CJK}                   
\usepackage{CJKfntef}        
\usepackage{fancyhdr}               
\usepackage{graphicx}                 
\usepackage{lastpage}    
\usepackage{listings}
\usepackage{fancybox}
\usepackage{xcolor}
\usepackage{fontspec}
\usepackage{layout}
\usepackage{titletoc}
\usepackage{listings}
\usepackage{color}
\usepackage{xcolor}
\usepackage{ctex}
\usepackage{mathrsfs}
\usepackage{hyperref}
\lstset{
	basicstyle={      
		\color{black}
		\fontspec{Consolas}
	},
	keywordstyle={
		\color{blue}
		\fontspec{Consolas}
	},
	numberstyle={
		\color{black}
		\textbf
	},
	rulecolor=\color{blue},
	numbers=left,                               
	frame=single,                            
	frameround=tttt,
	morekeywords={Sample, Input, Output},   % 可以手动添加关键字
}
\setmonofont{Consolas}
\newcommand{\stress}[1]{\textbf{\CJKunderdot{#1}}}

\definecolor{mys}{rgb}{1,0.2,0}%一个颜色
\definecolor{codegray}{rgb}{0.5,0.5,0.5}
\lstset{
	numberstyle=\ttfamily\color{codegray}
}

\linespread{1.5}%修改行距

% \titleformat{\subsection}{\fontsize{13}{16}}

% \begin{lstlisting}
	% \end{lstlisting}

%%%my packages
\usepackage{hyperref}% Links
\hypersetup{
	colorlinks=true,
	linkcolor=mys,
	urlcolor=blue
}
% \usepackage{pseudo}% Pseudo Code

\title{标题(无实际意义)}
\author{作者(无实际意义)}
\date{日期时间(无实际意义)}

\begin{document}
	\fontsize{12pt}{12pt}\selectfont
	
	\newpage
	\pagestyle{fancy}
	\lhead{\footnotesize \songti{信息学一周解题报告汇总\ \ \ -LSH}}
	\cfoot{\footnotesize 第 \thepage \ 页\qquad  共  \pageref{LastPage} 页}
	
	\phantomsection
	\addcontentsline{toc}{section}{}
	\rhead{\footnotesize \songti{第九周\ \ \ 2022.11.6 $\sim$ 2022.11.12}}
	
	
	\section*{【Goobye XinChou】马超战潼关}
	
	\phantomsection
	\addcontentsline{toc}{subsection}{}
	\subsection*{【题目描述】}
	
	https://uoj.ac/problem/704。
	
	给定一张 $S$ 连到所有左部点、左部点部分连向部分右部点、右部点全部连向 $T$ 的边不带权的二分图,问它的边最小割方案数。左右侧点数均 $\le 46$。
	
	\phantomsection
	\addcontentsline{toc}{subsection}{}
	\subsection*{【题解】}
	
	首先第一个观察就相当难以理解,求出一组最大匹配,每个匹配对应着三条边,这三条边中只会选一条,且选出的一定是匹配中的边。
	
	记匹配的三条边分别为 $0,1,2$,用 $a_u=0/1/2$ 表示这个匹配选出的那条边。
	
	而对于不在匹配中的边:
	
	\begin{itemize}
		\item 若两端点分别维护匹配 $u,v$ 中,则推出 $a_u=0$ 或者 $a_v=2$。
		\item 若 $u$ 位于匹配中,则 $a_u=0$。
		\item 若 $v$ 位于匹配中,则 $a_v=2$。
	\end{itemize}

	那么不难发现,只考虑 $1,2$ 构成的图,$1$ 可以是其中所有 $0$ 度点且彼此独立,方案数为 $2$ 的零度点个数次方。借助折半搜索和 fmt 即可做到 $O(2^{\frac n2}\cdot n^2)$。
	
	\phantomsection
	\addcontentsline{toc}{subsection}{}
	\subsection*{【提交记录】}
	
	https://uoj.ac/submission/591116(一个 80 分的搜索)。
	
	% -----------------------------------------------------------------------------------
	
	
	\section*{【清华集训 2015】静态仙人掌}
	
	\phantomsection
	\addcontentsline{toc}{subsection}{}
	\subsection*{【题目描述】}
	
	https://uoj.ac/problem/158。
	
	给定一棵有根的仙人掌,每个点上有 $0$ 或者 $1$,三种操作:
	
	\begin{itemize}
		\item [1.] 翻转一个点到根的最短简单路。
		\item [2.] 翻转一个点到根的最长简单路。
		\item [3.] 子树求和。
	\end{itemize}
	
	\phantomsection
	\addcontentsline{toc}{subsection}{}
	\subsection*{【题解】}
	
	最简单的做法:直接求出每个点对应的两种路径的 bitset,即可做到 $O(\frac {n^2}{w})$。
	
	复杂做法:把点简单分类一下,利用树剖可以解决(某静态的超仙人掌)。
	
	\phantomsection
	\addcontentsline{toc}{subsection}{}
	\subsection*{【提交记录】}
	
	https://uoj.ac/submission/590995。
	
	% -----------------------------------------------------------------------------------
	
	\section*{【清华集训 2015】小Q与找茬}
	
	\phantomsection
	\addcontentsline{toc}{subsection}{}
	\subsection*{【题目描述】}
	
	https://uoj.ac/problem/157。
	
	平面上有 $n$ 个点,多次询问矩形,你每次需要输出尽可能多的矩形内的点,不要求全部输出。
	
	\phantomsection
	\addcontentsline{toc}{subsection}{}
	\subsection*{【题解】}
	
	range-tree 加分散层叠可以轻松通过;事实上由于数据难以构造等原因,平面分块可以相当简洁且优秀。
	
	\phantomsection
	\addcontentsline{toc}{subsection}{}
	\subsection*{【提交记录】}
	
	https://uoj.ac/submission/590983。
	
	% -----------------------------------------------------------------------------------
	
	\section*{【Goobye XinChou】黄忠庆功宴}
	
	\phantomsection
	\addcontentsline{toc}{subsection}{}
	\subsection*{【题目描述】}
	
	https://uoj.ac/problem/705。
	
	给定一个数列,多次询问,每次询问给定 $x,l,k$,问 $\sum_{i=0}^{l-1}a_{(x+ik)\bmod n}$。
	
	\phantomsection
	\addcontentsline{toc}{subsection}{}
	\subsection*{【题解】}
	
	对于 $k\le n^{0.5}$,我们可以预处理出所有的公差的前缀和。
	
	对于其它的,我们整体考虑。设 $id=x+ik$,则 $i=\frac{id-x}k$,即 $id$ 会计入贡献的条件为 $(id-x)\cdot k^{-1}\bmod n < l$。
	
	注意到每 $\frac{n}{k^{-1}}$ 会分一段,则一共会划分成 $O(k^{-1})$ 个公差为 $1$ 的等差数列。
	
	我们把上面的两个做法结合一下,设 $k=\frac xy$,一定可以使 $x,|y|\le n^{0.5}$,对于每个 $x$ 先预处理出前缀和,然后询问的时候 $O(y)$ 询问即可。
	
	\phantomsection
	\addcontentsline{toc}{subsection}{}
	\subsection*{【提交记录】}
	
	https://uoj.ac/submission/590974。
	
	% -----------------------------------------------------------------------------------
	
	\section*{【清华集训 2014】简单回路}
	
	\phantomsection
	\addcontentsline{toc}{subsection}{}
	\subsection*{【题目描述】}
	
	https://uoj.ac/problem/39。
	
	在一个有障碍点的 $n$ 行 $m$ 列的网格图中,如果该网格图中的回路满足下面两个条件:
	
	\begin{itemize}
		\item 不经过任何一个障碍点
		\item 回路不自交
	\end{itemize}
	
	则我们称该回路为合法的简单回路。
	
	现在有 $Q$ 个询问,每次询问有多少条合法的简单回路经过了 $(x,y)$ 与 $(x+1,y)$ 之间的边。
	
	\phantomsection
	\addcontentsline{toc}{subsection}{}
	\subsection*{【题解】}
	
	进行一次前缀插头 dp 和一次后缀插头 dp,询问的时候暴力合并即可。主要是收获了一些精细实现插头 dp 的经验。
	
	\phantomsection
	\addcontentsline{toc}{subsection}{}
	\subsection*{【提交记录】}
	
	https://uoj.ac/submission/590670。
	
	% -----------------------------------------------------------------------------------
	
	\section*{【清华集训 2014】玄学}
	
	\phantomsection
	\addcontentsline{toc}{subsection}{}
	\subsection*{【题目描述】}
	
	https://uoj.ac/problem/46。
	
	一个序列,若干个操作,强制在线:
	
	\begin{itemize}
		\item [1.] 新增一个修改,它会给区间 $[l,r]$ 内的所有点带来不可逆、不可交换、有结合律的一个操作。
		\item [2.] 询问如果一次执行操作 $[l,r]$,序列上的 $k$ 位置会被改成什么。
	\end{itemize}
	
	\phantomsection
	\addcontentsline{toc}{subsection}{}
	\subsection*{【题解】}
	
	其实只要维护一个对时间的线段树,线段树上的每个结点上的所有操作会把序列划分成若干个区间,区间内的所有数相同。我们只要记下区间的所有左右端点即可。
	
	随着线段树逐渐被填充,层层向上归并,并分散层叠,即可做到完全的单 $\log$ 时空。
	
	\phantomsection
	\addcontentsline{toc}{subsection}{}
	\subsection*{【提交记录】}
	
	https://uoj.ac/submission/590475。
	
	% -----------------------------------------------------------------------------------
	
	\section*{克莱茵蓝彼岸花}
	
	\phantomsection
	\addcontentsline{toc}{subsection}{}
	\subsection*{【题目描述】}
	
	https://loj.ac/p/6786。
	
	千寻、眠雪给了你 $n$ 个「书本」,每个「书本」有四个「属性」$a_i,b_i,c_i,d_i$.
	
	形式化地来说,你需要将「书本」划分为三个**非空**集合 $S_1,S_2,S_3$,**最小化**目标函数 $\mathbf{\Xi}$,定义为:
	
	$$
	\mathbf{\Xi}=\left(\max_{x=1}^3\max_{i\in S_x}a_i\right)\times\left(\sum_{x=1}^3\max_{i\in S_x}b_i\right)\times\left(\max_{x=1}^3\sum_{i\in S_x}c_i\right)\times\left(\sum_{x=1}^3\sum_{i\in S_x}d_i\right)
	$$
	
	求出目标函数的**最小**可能值.
	
	\phantomsection
	\addcontentsline{toc}{subsection}{}
	\subsection*{【题解】}
	
	首先按照 $b$ 排序,相当于把序列分成三段,第一段全部属于 $S_1$,第二段属于 $S_1$ 或 $S_2$,第三段属于 $S_1$ 或 $S_2$ 或 $S_3$。
	
	优化枚举,首先枚举第三段的长度,从短到长枚举,顺带维护一个二维 $0/1$ 背包;然后从短到长枚举第二段的长度,维护一个背包,然后利用 bitset 等来优化,再加上各种各样的剪枝,可以通过。
	
	\phantomsection
	\addcontentsline{toc}{subsection}{}
	\subsection*{【提交记录】}
	
	https://loj.ac/s/1628362(高居最优解)。
	
\end{document}


















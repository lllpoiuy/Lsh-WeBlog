\documentclass[UTF8,12pt,a4paper]{ctexart}
\usepackage[utf8]{inputenc}
\usepackage{color}
\usepackage{array}
\usepackage{diagbox}
\usepackage{multicol}
\usepackage{multirow}
\usepackage{CJKfntef}
\usepackage{booktabs}
\usepackage{fancyhdr}
\usepackage{graphicx}
\usepackage{lastpage}
\usepackage{indentfirst}
\usepackage{amsmath,amssymb}
\usepackage{tabu}
\usepackage[colorlinks,linkcolor=blue,anchorcolor=blue,citecolor=green,bookmarks=true]{hyperref}
%\usepackage[breaklinks,colorlinks,linkcolor=black,citecolor=black,urlcolor=black]{hyperref}
\usepackage{bookmark}
\usepackage{inconsolata}
\usepackage{geometry}
\usepackage{titlesec}
\usepackage{paralist}
\usepackage[ruled,linesnumbered]{Algorithm2e}
\setlength{\parindent}{2em}
\geometry{left=2.45cm,right=2.45cm,top=2.75cm,bottom=2.6cm}
\setcounter{secnumdepth}{0}
\let\itemize\compactitem
\let\enditemize\endcompactitem
\let\enumerate\compactenum
\let\endenumerate\endcompactenum
\let\description\compactdesc
\let\enddescription\endcompactdesc

\usepackage{booktabs} % for much better looking tables
\usepackage{array} % for better arrays (eg matrices) in maths
\usepackage{paralist} % very flexible & customisable lists (eg. enumerate/itemize, etc.)
\usepackage{verbatim} % adds environment for commenting out blocks of text & for better verbatim
\usepackage{subfig} % make it possible to include more than one captioned figure/table in a single float
\usepackage{xcolor}
\usepackage{endnotes}
\usepackage{multirow}
\usepackage{notoccite}
\usepackage{longtable}
\usepackage{geometry}
\usepackage{multicol}
\usepackage{multirow}
\usepackage{tabu}
\usepackage{xeCJK}
\usepackage{CJK}                   
\usepackage{CJKfntef}        
\usepackage{fancyhdr}               
\usepackage{graphicx}                 
\usepackage{lastpage}    
\usepackage{listings}
\usepackage{fancybox}
\usepackage{xcolor}
\usepackage{fontspec}
\usepackage{layout}
\usepackage{titletoc}
\usepackage{listings}
\usepackage{color}
\usepackage{xcolor}
\usepackage{ctex}
\usepackage{mathrsfs}
\usepackage{hyperref}
\lstset{
	basicstyle={      
		\color{black}
		\fontspec{Consolas}
	},
	keywordstyle={
		\color{blue}
		\fontspec{Consolas}
	},
	numberstyle={
		\color{black}
		\textbf
	},
	rulecolor=\color{blue},
	numbers=left,                               
	frame=single,                            
	frameround=tttt,
	morekeywords={Sample, Input, Output},   % 可以手动添加关键字
}
\setmonofont{Consolas}
\newcommand{\stress}[1]{\textbf{\CJKunderdot{#1}}}

\definecolor{mys}{rgb}{1,0.2,0}%一个颜色
\definecolor{codegray}{rgb}{0.5,0.5,0.5}
\lstset{
	numberstyle=\ttfamily\color{codegray}
}

\linespread{1.5}%修改行距

% \titleformat{\subsection}{\fontsize{13}{16}}

% \begin{lstlisting}
	% \end{lstlisting}

%%%my packages
\usepackage{hyperref}% Links
\hypersetup{
	colorlinks=true,
	linkcolor=mys,
	urlcolor=blue
}
% \usepackage{pseudo}% Pseudo Code

\title{标题(无实际意义)}
\author{作者(无实际意义)}
\date{日期时间(无实际意义)}

\begin{document}
	\fontsize{12pt}{12pt}\selectfont
	
	\newpage
	\pagestyle{fancy}
	\lhead{\footnotesize \songti{信息学一周解题报告汇总\ \ \ -LSH}}
	\cfoot{\footnotesize 第 \thepage \ 页\qquad  共  \pageref{LastPage} 页}
	
	\phantomsection
	\addcontentsline{toc}{section}{}
	\rhead{\footnotesize \songti{第五、六周\ \ \ 2022.10.9 $\sim$ 2022.10.22}}
	
	\section*{正睿 OI 好题精选}

	\phantomsection
	\addcontentsline{toc}{subsection}{}
	\subsection*{22 冲刺 day1-密码}
	
	http://www.zhengruioi.com/contest/1268/problem/2356。
	
	精妙的构造题!
	
	\phantomsection
	\addcontentsline{toc}{subsection}{}
	\subsection*{22 冲刺 day1-旅行}
	
	http://www.zhengruioi.com/contest/1268/problem/2355。
	
	单调栈维护,对于每种余数分别建单调栈,每种余数下每个点需要代表一个整块,需要再写一个单调栈来维护整块;散块直接在单调栈上二分即可。复杂度 $O(n\log n)$。
	
	\phantomsection
	\addcontentsline{toc}{subsection}{}
	\subsection*{22 冲刺 day1-狼人}
	
	http://www.zhengruioi.com/contest/1268/problem/2354。
	
	对于每种颜色分别 dp,在 dp 的时候剪枝好上下界。注意到 $f_{u,i}$ 形式的 dp,当 $i\le \min(k,sz_u)$ 时,复杂度位 $O(kn)$;于是总复杂度为 $O(n\sum k)=O(n^2)$。
	
	\phantomsection
	\addcontentsline{toc}{subsection}{}
	\subsection*{22 冲刺 day2-Permutation}
	
	http://www.zhengruioi.com/contest/1269/problem/2390。
	
	对于只出现一次的素数,就不要用了;出现两次的素数,最多使用两个,因为 $1p 2p 2x .... 2y 2q 1q$ 就已经把数列限制死了。对于出现三次的素数,我们可以 $2p 1p 3p 3p 1p 2p$ 配对起来,多出来的一个找 $12$ 帮助配对一下;其它的素数都可以自成一派 $4p ..... 2p$。
	
	\phantomsection
	\addcontentsline{toc}{subsection}{}
	\subsection*{22 冲刺 day3-等差数列}
	
	http://www.zhengruioi.com/contest/1270/problem/2400。
	
	考虑公比是 $k$,那么答案是 $O(n^2k)$ 级别的,而一个显然的代价上界是 $\sum a$,所以有效的 $k$ 只有 $O(\frac{\max a}{n})+O(1)$ 级别。直接枚举!
	
	如何计算对应的最小代价呢?我们另每个数都减去 $ik$,那么就是要另前后分别相等;考虑怎么计算一个前缀它要都相等要花费的最小代价,这个东西分奇数位和偶数位分贝是凸的,用 priority\_queue 维护即可。
	
	\phantomsection
	\addcontentsline{toc}{subsection}{}
	\subsection*{22 冲刺 day4-简单题}
	
	http://www.zhengruioi.com/contest/1271/problem/2426。
	
	考虑像笛卡尔树那样枚举最大值在哪里,并且这个最大值可以放缩,不需要一定是真正的最大值。转移也就变成了在凸包上查询的形式,可以离线双指针 $n^2$ 次,也可以直接在凸包上二分。
	
	\phantomsection
	\addcontentsline{toc}{subsection}{}
	\subsection*{22 冲刺 day4-数数题}
	
	http://www.zhengruioi.com/contest/1271/problem/2427。
	
	一个点的期望深度容易通过 dp 简单算出,问题变为求 $E(\mathrm{dep_{\operatorname{lca}(u,v)}}$
	考虑枚举一个点 $l$,计算其作为 lca 的概率;
	
	$$
	\sum_{S_1\subseteq (l,x)}\sum_{S_2\subseteq (1,y)} [S_1\cap S_2=\emptyset]\cdot \left(\prod_{u\in S_1} \frac {a_u}{b_u}\right)\cdot \left(\prod_{u\in S_2} \frac {a_u}{b_u}\right)\cdot\frac {a_l^2}{b_xb_y}
	$$
	
	结合类似生成函数的思想可以得到:
	
	$$
	\frac {a_l^2}{b_xb_y}\prod_{i\in(l,x)}(1+2c_u)\prod_{i\in(x,y)} (1+c_u)
	$$
	
	这就特别好维护了。
	
	\phantomsection
	\addcontentsline{toc}{subsection}{}
	\subsection*{22 冲刺 day5-排列}
	
	http://www.zhengruioi.com/contest/1275/problem/2440。
	
	提取出直径,分支上的情况数非常少;但是由于直径有 $O(n^2)$ 条,难以去重,所有对分支直接搜索,总复杂度还是 $O(\operatorname{poly}(n))$。
	
	\phantomsection
	\addcontentsline{toc}{subsection}{}
	\subsection*{22 冲刺 day5-树}
	
	http://www.zhengruioi.com/contest/1275/problem/2439。
	
	依然是提取出直径,然后考虑这个直径的中点,一定在一条边上徘徊。
	
	\phantomsection
	\addcontentsline{toc}{subsection}{}
	\subsection*{22 冲刺 day6-脉冲星}
	
	http://www.zhengruioi.com/contest/1276/problem/2372。
	
	事实上不等于 $R$ 的数可以达到 $\log$ 个。所以我们直接 dp,dp 这 log 个中有多少个顶着上界、多少个顶着下界,然后从高到低位考虑即可。$O(\log^4)$。
	
	用贪心可以优化到 $O(\log^2)$。
	
	\phantomsection
	\addcontentsline{toc}{subsection}{}
	\subsection*{22 冲刺 day6-括号序列}
	
	http://www.zhengruioi.com/contest/1276/problem/2371。
	
	神秘带悔贪心。
	
	\phantomsection
	\addcontentsline{toc}{subsection}{}
	\subsection*{22 冲刺 day6-脉冲星}
	
	http://www.zhengruioi.com/contest/1276/problem/2372。
	
	事实上不等于 $R$ 的数可以达到 $\log$ 个。所以我们直接 dp,dp 这 log 个中有多少个顶着上界、多少个顶着下界,然后从高到低位考虑即可。$O(\log^4)$。
	
	用贪心可以优化到 $O(\log^2)$。
	
	\phantomsection
	\addcontentsline{toc}{subsection}{}
	\subsection*{22 冲刺 day7-D}
	
	http://www.zhengruioi.com/contest/1277/problem/2364。
	
	建出可持久化线段树维护 Trie 跑匹配 dp,可以发现复杂度可以过到 70;正解则是直接在原图上用可持久化的思想来把非法路径容斥掉。
	
	\phantomsection
	\addcontentsline{toc}{subsection}{}
	\subsection*{22 冲刺 day7-C}
	
	http://www.zhengruioi.com/contest/1277/problem/2363。
	
	哈密顿回路最长一定是每条边的贡献次数都是其两边的点数取 $\min$ 乘二!
	
	一种方式是枚举重心,然后要求每个孩子都小于等于一半,但是只能做到 $O(nk)$。
	
	正解是直接考虑一个点两侧的点数来 dp,通过凸性等等优化。
	
	\phantomsection
	\addcontentsline{toc}{subsection}{}
	\subsection*{22 冲刺 day8-序列}
	
	http://www.zhengruioi.com/contest/1278/problem/2368。
	
	考虑 $1$ 出现的位置的奇偶性,相邻的两个相同的可以合并,所以最后剩下的一定是奇偶奇偶奇偶...,这个有结合律,可以用线段树维护。
	
	\phantomsection
	\addcontentsline{toc}{subsection}{}
	\subsection*{22 冲刺 day8-树}
	
	http://www.zhengruioi.com/contest/1278/problem/2367。
	
	硬猜结论,结合拓展凯莱定理来做!
	
\end{document}


















\documentclass[UTF8,12pt,a4paper]{ctexart}
\usepackage[utf8]{inputenc}
\usepackage{color}
\usepackage{array}
\usepackage{diagbox}
\usepackage{multicol}
\usepackage{multirow}
\usepackage{CJKfntef}
\usepackage{booktabs}
\usepackage{fancyhdr}
\usepackage{graphicx}
\usepackage{lastpage}
\usepackage{indentfirst}
\usepackage{amsmath,amssymb}
\usepackage{tabu}
\usepackage[colorlinks,linkcolor=blue,anchorcolor=blue,citecolor=green,bookmarks=true]{hyperref}
%\usepackage[breaklinks,colorlinks,linkcolor=black,citecolor=black,urlcolor=black]{hyperref}
\usepackage{bookmark}
\usepackage{inconsolata}
\usepackage{geometry}
\usepackage{titlesec}
\usepackage{paralist}
\usepackage[ruled,linesnumbered]{Algorithm2e}
\setlength{\parindent}{2em}
\geometry{left=2.45cm,right=2.45cm,top=2.75cm,bottom=2.6cm}
\setcounter{secnumdepth}{0}
\let\itemize\compactitem
\let\enditemize\endcompactitem
\let\enumerate\compactenum
\let\endenumerate\endcompactenum
\let\description\compactdesc
\let\enddescription\endcompactdesc

\usepackage{booktabs} % for much better looking tables
\usepackage{array} % for better arrays (eg matrices) in maths
\usepackage{paralist} % very flexible & customisable lists (eg. enumerate/itemize, etc.)
\usepackage{verbatim} % adds environment for commenting out blocks of text & for better verbatim
\usepackage{subfig} % make it possible to include more than one captioned figure/table in a single float
\usepackage{xcolor}
\usepackage{endnotes}
\usepackage{multirow}
\usepackage{notoccite}
\usepackage{longtable}
\usepackage{geometry}
\usepackage{multicol}
\usepackage{multirow}
\usepackage{tabu}
\usepackage{xeCJK}
\usepackage{CJK}                   
\usepackage{CJKfntef}        
\usepackage{fancyhdr}               
\usepackage{graphicx}                 
\usepackage{lastpage}    
\usepackage{listings}
\usepackage{fancybox}
\usepackage{xcolor}
\usepackage{fontspec}
\usepackage{layout}
\usepackage{titletoc}
\usepackage{listings}
\usepackage{color}
\usepackage{xcolor}
\usepackage{ctex}
\usepackage{mathrsfs}
\usepackage{hyperref}
\lstset{
	basicstyle={      
		\color{black}
		\fontspec{Consolas}
	},
	keywordstyle={
		\color{blue}
		\fontspec{Consolas}
	},
	numberstyle={
		\color{black}
		\textbf
	},
	rulecolor=\color{blue},
	numbers=left,                               
	frame=single,                            
	frameround=tttt,
	morekeywords={Sample, Input, Output},   % 可以手动添加关键字
}
\setmonofont{Consolas}
\newcommand{\stress}[1]{\textbf{\CJKunderdot{#1}}}

\definecolor{mys}{rgb}{1,0.2,0}%一个颜色
\definecolor{codegray}{rgb}{0.5,0.5,0.5}
\lstset{
	numberstyle=\ttfamily\color{codegray}
}

\linespread{1.5}%修改行距

% \titleformat{\subsection}{\fontsize{13}{16}}

% \begin{lstlisting}
	% \end{lstlisting}

%%%my packages
\usepackage{hyperref}% Links
\hypersetup{
	colorlinks=true,
	linkcolor=mys,
	urlcolor=blue
}
% \usepackage{pseudo}% Pseudo Code

\title{标题(无实际意义)}
\author{作者(无实际意义)}
\date{日期时间(无实际意义)}

\begin{document}
	\fontsize{12pt}{12pt}\selectfont
	
	\newpage
	\pagestyle{fancy}
	\lhead{\footnotesize \songti{信息学一周解题报告汇总\ \ \ -LSH}}
	\cfoot{\footnotesize 第 \thepage \ 页\qquad  共  \pageref{LastPage} 页}
	
	\phantomsection
	\addcontentsline{toc}{section}{}
	\rhead{\footnotesize \songti{第七周\ \ \ 2022.10.30 $\sim$ 2022.11.5}}
	
	
	\section*{【清华集训 2014】玛里苟斯}
	
	\phantomsection
	\addcontentsline{toc}{subsection}{}
	\subsection*{【题目描述】}
	
	https://uoj.ac/problem/36。
	
	$S$ 是一个可重集合,$S=\{a_1,a_2,\cdots,a_n\}$。
	
	等概率随机取 $S$ 的一个子集 $A$。
	
	计算出 $A$ 中所有元素异或和 $x$, 求 $x^k$ 的期望。
	
	$n\le 10^5,k\le 5$,保证答案小于 $2^63$,要求输出精确小数答案。
	
	\phantomsection
	\addcontentsline{toc}{subsection}{}
	\subsection*{【题解】}
	
	一种脑子有问题的方法是:把 $x^k$ 拆开来:
	
	$$
	\begin{aligned}
		x^k
		&= \sum_{i_1,i_2,\cdots, i_k} \prod_{j=1}^k \left(x \operatorname{and} 2^{i_j}\right)\\
		&= \sum_{i_1,i_2,\cdots, i_k} 2^{\sum_{j=1}^k i_j} \prod_{j=1}^k\left[x \operatorname{and} 2^{i_j} \neq 0\right]
	\end{aligned}
	$$
	
	枚举了 $i$ 后,问题就变成了求这些位上都等于 $1$ 的概率,也就可以把所有的数只保留 $i$ 中的位。对于所有相同的数,显然只要保留一个!因为恰好有一半的情况里它的贡献是 $1$!
	
	根据答案小于 $2^63$,可以算出复杂度是 $O\left(\min(n,\frac{63}k)\cdot\binom{\frac{63}{k}+k-1}{k}\right)$,通过精湛的卡常可以通过。
	
	事实上,子集异或的事情只要不是集合幂级数学傻了,就应该意识到每个在其线性基内的数的出现次数都等于 $2$ 的自由元个数次方!对于 $k\ge 3$ 采取暴力遍历线性基即可。
	
	\phantomsection
	\addcontentsline{toc}{subsection}{}
	\subsection*{【提交记录】}
	
	https://uoj.ac/submission/590211。
	
	% -----------------------------------------------------------------------------------
	
	
	\section*{【清华集训 2014】主旋律}
	
	\phantomsection
	\addcontentsline{toc}{subsection}{}
	\subsection*{【题目描述】}
	
	https://uoj.ac/problem/37。
	
	给定一张有向图,求其强连通生成子图个数。$n\le 15$。
	
	\phantomsection
	\addcontentsline{toc}{subsection}{}
	\subsection*{【题解】}
	
	强连通的条件是,缩点后,不存在一个 $0$ 入度的非全集的强连通分量,也就是这个强连通分量入度为 $0$ 的集合是空集。
	
	我们直接容斥这个入度为 $0$ 的集合,记 $f_S$ 表示 $S$ 导出子图有多少个生成子图强连通,$g_S$ 表示把 $S$ 拆分成 $T_1,T_2,\cdots,T_k$ 的 $(-1)^k\cdot \prod f_{T_i}$ 的和,然后就能自然转移。
	
	求两个集合之间的边数,可以利用 bitmask 做到总复杂度 $O(3^n\cdot n)$。
	
	\phantomsection
	\addcontentsline{toc}{subsection}{}
	\subsection*{【提交记录】}
	
	https://uoj.ac/submission/590164。
	
	% -----------------------------------------------------------------------------------
	
	\section*{【清华集训 2014】虫逢}
	
	\phantomsection
	\addcontentsline{toc}{subsection}{}
	\subsection*{【题目描述】}
	
	https://uoj.ac/problem/45。
	
	给定一个基因集合(大小为 $M$),$N$ 个生物,每个生物都有 $L$ 基因(\textbf{保证数据随机选取})。然后每个生物都分裂成了两个生物,一个保留原有基因,一个有恰好一半的基因被随机替换为了它\textbf{本来没有}的基因。现在把这 $2N$ 个生物打乱告诉你,你要帮忙两两配对。$N=M=L^2\in [400,16900]$。
	
	\phantomsection
	\addcontentsline{toc}{subsection}{}
	\subsection*{【题解】}
	
	本题读题难度非常大,由于数据完全随机且 $L$ 集合相当大,可以视为两两配对是确定的,不存在一个一般图最大匹配的问题!
	
	暴力 check 肯定是要 TLE 的,那这时的思路肯定是用一些 Hash 之类的信息来先划分等价类,然后只 check 等价类之内的信息,以减少 check 次数。
	
	我们给基因库随机一个排列,令每个生物的值为它的基因里最靠前的一个,那么同源的两个生物值相等的概率是 $\frac 13$,不同源的值相同的概率是 $\frac 1L$。这还不够,再给它加上双哈希,冲撞率进一步降低,每次可以将 $N$ 降低 $\frac 19$。可以通过。
	
	\phantomsection
	\addcontentsline{toc}{subsection}{}
	\subsection*{【提交记录】}
	
	https://uoj.ac/submission/590159。
	
	% -----------------------------------------------------------------------------------
	
	\section*{【CF1225G】To Make 1}
	
	\phantomsection
	\addcontentsline{toc}{subsection}{}
	\subsection*{【题目描述】}
	
	https://www.luogu.com.cn/problem/CF1225G。
	
	合并 $n(n\le 16)$ 个整数为 $1$ 个,合并 $x$ 和 $y$ 时将会删除 $x$ 和 $y$ 然后加入一个整数 $f(x+y)$,其中
	
	$$
	f(x)=\left\{ \begin{aligned} &f\left(\small \frac xk\right) , k|x \\&x,\text{otherwise} \end{aligned}\right.
	$$
	
	如果最后留下的数字可以为 $1$,输出 YES 并输出方案(每次是哪两个数字合并)。否则输出 NO。保证 $k \nmid a_i$ 且 $\sum a_i \le 2000$。
	
	\phantomsection
	\addcontentsline{toc}{subsection}{}
	\subsection*{【题解】}
	
	事实上,我们就是要寻找一组 $b_{1\sim n}$,使得 $\sum a_i\cdot k^{-b_i}=1$。
	
	证明(充分):对于每种合法方案,可以还原出一组系数 $b$。
	
	证明(必要):从 $b$ 中可以递归地还原出方案,因为 $k\nmid a_i$,而根据 $k^{\max b}=\sum_{i=1}^n a_i^{\max b - b_i}$,$b_i=\max b_i$ 的数不可能只有一个,否则等式两边模 $k$ 的值一定不同!所以可以合并这两个,然后递归下去构造。
	
	如果改成选 $k$ 个数,合并成和除 $k$ 的话,那其实就再加上一个限制条件 $\sum k^{-b_i}=1$。
	
	那剩下的 dp 其实就相当简单了,设 $f_{S, i}$ 表示合并了集合 $S$ 和为 $i$ 是否可以,一种是 $f_{S}$ 向 $f_{S\cup u}$ 转移,一种是 $f_S$ 内部全部尽可能除以 $k$ 的转移。用 bitset 优化即可,输出方案就按照上面证明的那样去递归构造。
	
	复杂度 $O\left(\frac{n\cdot 2^n\cdot \sum a}{w}+2^n\cdot \sum a \right)$。
	
	\phantomsection
	\addcontentsline{toc}{subsection}{}
	\subsection*{【提交记录】}
	
	https://www.luogu.com.cn/record/92669335。
	
	% -----------------------------------------------------------------------------------
	
	\section*{【CF1149D】Abandoning Roads}
	
	\phantomsection
	\addcontentsline{toc}{subsection}{}
	\subsection*{【题目描述】}
	
	https://www.luogu.com.cn/problem/CF1149D。
	
	一张 $n$ 个点 $m$ 条边的无向图,只有 $a,b$ 两种边权($a<b$),对于每个 $i$,求图中所有的最小生成树中,从 $1$ 到 $i$ 距离的最小值。$n\le 70,m\le 200$。
	
	\phantomsection
	\addcontentsline{toc}{subsection}{}
	\subsection*{【题解】}
	
	只保留 $a$ 的边,可以先求出连通块和连通块内部的全源最短路,然后再考虑上 $b$ 的边去找 $1$ 到 $i$ 的最短路,但是要加上一个限制——不同两次进入同一个连通块。
	
	我的解决方案是 A*,拿下最优解。
	
	更为靠谱的方案是,会重复经过的连通块,至少有 $4$ 个点,所以只状压这些连通块,复杂度 $2^{\frac n4}$ 左右。
	
	\phantomsection
	\addcontentsline{toc}{subsection}{}
	\subsection*{【提交记录】}
	
	https://www.luogu.com.cn/record/92653970。
	
	% -----------------------------------------------------------------------------------
	
	\section*{【CF1149E】Election Promises}
	
	\phantomsection
	\addcontentsline{toc}{subsection}{}
	\subsection*{【题目描述】}
	
	https://www.luogu.com.cn/problem/CF1149E。
	
	在 Byteland,两个党派——WA 党和 TLE 党,他们都在不停地承诺降税。
	
	Byteland 有 $n$ 个城市,$m$ 条单行道,并且不能从一个城市出发又回到这个城市。竞选开始前,第 $i$ 个城市的税收是非负整数$h_i$。
	
	一次拉票的过程:这个党派会来到一个城市 $x$,承诺会将这个城市的税收**降低**为一个非负整数。同时,这个党派可以将从 $x$ 走一条单行道可以直达的城市的税收做任意更改,变高变低都行,但还得保证该城市的税收是非负整数。
	
	两党交替拉票,无法执行一次拉票的党派会输。问胜负。
	
	\phantomsection
	\addcontentsline{toc}{subsection}{}
	\subsection*{【题解】}
	
	求出原图上每个点的 $SG$ 值,如果对于每个 $x$,都有 $\oplus_{u=1}^n[SG_u = x]h_u=0$,那么先手必败;否则先手必胜。
	
	证明就相当容易了,找到最大的异或和不为 $0$ 的 $x$,它一定能把所有的 $x$ 都一并改成 $0$。
	
	\phantomsection
	\addcontentsline{toc}{subsection}{}
	\subsection*{【提交记录】}
	
	https://www.luogu.com.cn/record/92602582。
	
	% -----------------------------------------------------------------------------------
	
	\section*{【CF1698G】Long Binary String}
	
	\phantomsection
	\addcontentsline{toc}{subsection}{}
	\subsection*{【题目描述】}
	
	https://www.luogu.com.cn/problem/CF1698G。
	
	现有一个无穷大的 $01$ 序列,初始全为 $0$。给定一个有限的 $01$ 序列 $S$,每次操作可以将原序列一个长为 $|S|$ 的子段异或上  $S$ ,最终需要使得整个序列只有两个 $1$。
	
	问最终字典序最大的序列的两个 $1$ 分别所处的位置。$|S| \le 35$。
	
	\phantomsection
	\addcontentsline{toc}{subsection}{}
	\subsection*{【题解】}
	
	左边那个 $1$ 显然是 $\operatorname{lowbit}(S)$,在此之后显然可以把 $S$ 的低位的 $0$ 全部去掉。
	
	显然一个位置只会被异或上最多一次,把每个位置是否要异或记录下来为 $F$ 数组,那实际上我们就可以把这个过程看作是一个生成函数的乘法!也就是求最小的 $k$ 使得:
	
	$$
	x^k+1=S(x)\cdot F(x)
	$$
	
	不妨视作:
	
	$$
	x^k=1 \pmod{S}
	$$
	
	BSGS 即可!
	
	\phantomsection
	\addcontentsline{toc}{subsection}{}
	\subsection*{【提交记录】}
	
	https://www.luogu.com.cn/record/92599215。
	
	% -----------------------------------------------------------------------------------
	
	\section*{【CF1257G】Divisor Set}
	
	\phantomsection
	\addcontentsline{toc}{subsection}{}
	\subsection*{【题目描述】}
	
	https://www.luogu.com.cn/problem/CF1257G。
	
	给一个数 $m$ 的质因子分解 $m=\prod_{i=1}^np_i$,找出一个最大的集合 $S$ 满足 $\forall x\in S,\,x\mid m$ 且 $\forall x,\,y\in S,\,x\nmid y\;(x\neq y)$ 。
	
	输出 $|S|\bmod 998244353$ 。
	
	\phantomsection
	\addcontentsline{toc}{subsection}{}
	\subsection*{【题解】}
	
	当每个素数两两不同的时候,显然是取 $\left\lfloor\frac n2\right\rfloor$ 个最优。
	
	猜想:一般情况下仍然是取 $\left\lfloor\frac n2\right\rfloor$ 个最优,再去重。
	
	为了计算这种策略方案数,可以用一个分治 NTT 来计算。
	
	\phantomsection
	\addcontentsline{toc}{subsection}{}
	\subsection*{【提交记录】}
	
	https://www.luogu.com.cn/record/92546117。
	
	% -----------------------------------------------------------------------------------
	
	\section*{【ZJOI2015】地震后的幻想乡}
	
	\phantomsection
	\addcontentsline{toc}{subsection}{}
	\subsection*{【题目描述】}
	
	https://www.luogu.com.cn/problem/P3343。
	
	一张无向图,每条边的权值在 $[0,1]$ 之间随机,求最小生成树的最大边的大小的期望。$n\le 10$。
	
	\phantomsection
	\addcontentsline{toc}{subsection}{}
	\subsection*{【题解】}
	
	设 $P(x)$ 为最大边小于等于 $x$ 的概率,那么答案就是 $1-\int P(x)\texttt{d}x$。
	
	显然 $P(x)$ 是一个关于 $x^i\cdot (1-x)^{m-i}$ 的多项式,系数为 $i$ 条边能使得图连通的方案数。
	
	那么跑一个集合幂级数 $\ln$ 形状的 dp 就行了!
	
	\phantomsection
	\addcontentsline{toc}{subsection}{}
	\subsection*{【提交记录】}
	
	https://www.luogu.com.cn/record/92375444(次优解)。
	
\end{document}


















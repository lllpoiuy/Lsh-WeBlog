\documentclass[UTF8,12pt,a4paper]{ctexart}
\usepackage[utf8]{inputenc}
\usepackage{color}
\usepackage{array}
\usepackage{diagbox}
\usepackage{multicol}
\usepackage{multirow}
\usepackage{CJKfntef}
\usepackage{booktabs}
\usepackage{fancyhdr}
\usepackage{graphicx}
\usepackage{lastpage}
\usepackage{indentfirst}
\usepackage{amsmath,amssymb}
\usepackage{tabu}
\usepackage[colorlinks,linkcolor=blue,anchorcolor=blue,citecolor=green,bookmarks=true]{hyperref}
%\usepackage[breaklinks,colorlinks,linkcolor=black,citecolor=black,urlcolor=black]{hyperref}
\usepackage{bookmark}
\usepackage{inconsolata}
\usepackage{geometry}
\usepackage{titlesec}
\usepackage{paralist}
\usepackage[ruled,linesnumbered]{Algorithm2e}
\setlength{\parindent}{2em}
\geometry{left=2.45cm,right=2.45cm,top=2.75cm,bottom=2.6cm}
\setcounter{secnumdepth}{0}
\let\itemize\compactitem
\let\enditemize\endcompactitem
\let\enumerate\compactenum
\let\endenumerate\endcompactenum
\let\description\compactdesc
\let\enddescription\endcompactdesc

\usepackage{booktabs} % for much better looking tables
\usepackage{array} % for better arrays (eg matrices) in maths
\usepackage{paralist} % very flexible & customisable lists (eg. enumerate/itemize, etc.)
\usepackage{verbatim} % adds environment for commenting out blocks of text & for better verbatim
\usepackage{subfig} % make it possible to include more than one captioned figure/table in a single float
\usepackage{xcolor}
\usepackage{endnotes}
\usepackage{multirow}
\usepackage{notoccite}
\usepackage{longtable}
\usepackage{geometry}
\usepackage{multicol}
\usepackage{multirow}
\usepackage{tabu}
\usepackage{xeCJK}
\usepackage{CJK}                   
\usepackage{CJKfntef}        
\usepackage{fancyhdr}               
\usepackage{graphicx}                 
\usepackage{lastpage}    
\usepackage{listings}
\usepackage{fancybox}
\usepackage{xcolor}
\usepackage{fontspec}
\usepackage{layout}
\usepackage{titletoc}
\usepackage{listings}
\usepackage{color}
\usepackage{xcolor}
\usepackage{ctex}
\usepackage{mathrsfs}
\usepackage{hyperref}
\lstset{
	basicstyle={      
		\color{black}
		\fontspec{Consolas}
	},
	keywordstyle={
		\color{blue}
		\fontspec{Consolas}
	},
	numberstyle={
		\color{black}
		\textbf
	},
	rulecolor=\color{blue},
	numbers=left,                               
	frame=single,                            
	frameround=tttt,
	morekeywords={Sample, Input, Output},   % 可以手动添加关键字
}
\setmonofont{Consolas}
\newcommand{\stress}[1]{\textbf{\CJKunderdot{#1}}}

\definecolor{mys}{rgb}{1,0.2,0}%一个颜色
\definecolor{codegray}{rgb}{0.5,0.5,0.5}
\lstset{
	numberstyle=\ttfamily\color{codegray}
}

\linespread{1.5}%修改行距

% \titleformat{\subsection}{\fontsize{13}{16}}

% \begin{lstlisting}
	% \end{lstlisting}

%%%my packages
\usepackage{hyperref}% Links
\hypersetup{
	colorlinks=true,
	linkcolor=mys,
	urlcolor=blue
}
% \usepackage{pseudo}% Pseudo Code

\title{标题(无实际意义)}
\author{作者(无实际意义)}
\date{日期时间(无实际意义)}

\begin{document}
	\fontsize{12pt}{12pt}\selectfont
	
	\newpage
	\pagestyle{fancy}
	\lhead{\footnotesize \songti{信息学一周解题报告汇总\ \ \ -LSH}}
	\cfoot{\footnotesize 第 \thepage \ 页\qquad  共  \pageref{LastPage} 页}
	
	\phantomsection
	\addcontentsline{toc}{section}{}
	\rhead{\footnotesize \songti{第四周\ \ \ 2022.9.26 $\sim$ 2022.10.2}}
	
	行百里者半九十。
	
	
	\section*{【CF1696G】Fishingprince Plays With Array Again}
	
	\phantomsection
	\addcontentsline{toc}{subsection}{}
	\subsection*{【题目描述】}
	
	https://www.luogu.com.cn/problem/CF1696G。
	
	定义关于两个参数 $X$,$Y$ 的对于序列 $a$ 的函数 $f(a)$:
	
	你可以对序列从 $1$ 开始长度为 $n$ 的 $a$ 进行以下两种操作:($t$ 可以是小数)
	
	\begin{itemize}
		\item 在位置 $i\ (1\le i<n)$ 花费 $t$ 秒,使 $a_i\gets a_i-Xt,a_{i+1}\gets a_{i+1}-Yt$。
		\item 在位置 $i\ (1\le i<n)$ 花费 $t$ 秒,使 $a_i\gets a_i-Yt,a_{i+1}\gets a_{i+1}-Xt$。
	\end{itemize}
	
	$f(a)$ 为通过上述两个操作使得 $a$ 序列中所有元素小于等于 $0$ 所需要花费的最小时间。给定参数 $X$,$Y$ 与序列 $a$,每次进行下列两个操作中的一个:
	
	\begin{itemize}
		\item `1 k v` 将 $a_k$ 变为 $v$。
		\item `2 l r` 将 $a$ 序列中 $[l,r]$ 区间范围内的元素以原顺序取出作为序列 $b$,求 $f(b)$。
	\end{itemize}

	$n,q\le 2\times 10^5$。
	
	\phantomsection
	\addcontentsline{toc}{subsection}{}
	\subsection*{【题解】}
	
	设 $b_i$ 和 $c_i$ 表示 $i$ 位置处进行的两种操作的次数,那么可以列出其线性规划形式:
	
	$$
	\left\{
	\begin{aligned}
		&\operatorname{minimize} \sum b_i+\sum c_i\\
		&\forall i, Xb_{i-1}+Yb_i+Xc_i+Yc_{i-1}\ge a_i
	\end{aligned}
	\right.
	$$
	
	将其对偶,可以得到:
	
	$$
	\left\{
	\begin{aligned}
		&\operatorname{maximize} \sum d_i\cdot a_i\\
		&\forall i, Xd_{i-1}+Yd_i\le 1\\
		&\forall i, Yd_{i-1}+Xd_i\le 1\\
	\end{aligned}
	\right.
	$$
	
	由于线性规划的性质可以得出,$d$ 要么是 $0$,要么是 $\frac 1{X+Y}$,要么是 $1+\frac 1{\max(X,Y)}$,线段树维护矩阵连乘积即可。
	
	
	\phantomsection
	\addcontentsline{toc}{subsection}{}
	\subsection*{【提交记录】}
	
	https://www.luogu.com.cn/record/87864174。
	
	% -----------------------------------------------------------------------------------
	
	\section*{【CF1276F】Asterisk Substrings}
	
	\phantomsection
	\addcontentsline{toc}{subsection}{}
	\subsection*{【题目描述】}
	
	https://www.luogu.com.cn/problem/CF1276F。
	
	给定一个字符集为小写字母的长度为 $n$ 的字符串 $s$,设 $t_i$ 表示将 $s$ 中的第 $i$ 个字符替换为特殊字符 $\text{*}$ 时得到的字符串,比如当 $s=\text{abc}$ 时,$t_1=\text{*bc}$,$t_2 = \text{a*c}$,$t_3 = \text{ab*}$。
	
	求字符串集合 $\{s,t_1,t_2,t_3,...,t_n\}$ 中本质不同的子串个数(需要计算空串)。
	
	注意 $\text{*}$ 仅表示一个字符,不表示其他含义(如通配符)。$|s|\le 10^5$。
	
	\phantomsection
	\addcontentsline{toc}{subsection}{}
	\subsection*{【题解】}
	
	我们考虑有经过 * 的子串种类数量,设 $f(S,T)$ 表示 S*T 能否达到,不难发现 S 和 T 可以用后缀树或者前缀树上的结点来代替,那么 $f$ 的定义域大小就是 $n^2$ 级别的。
	
	枚举 * 的位置,每次相当于把后缀树和前缀树上分别的一条到根的路径上的点 $(u,v)$ 两两之间的 $f(u,v)$ 设为 $1$;我们最后就是要统计 $1$ 的个数。
	
	每个前缀树上的点,可以和它匹配的一定是一个后缀树的包含根的连通子树;我们考虑用启发式合并来维护这些子树的边权和——用 set 维护点集的 dfn 序,即可求出所有边的边权和。总复杂度 $O(n\log^2n)$。
	
	
	\phantomsection
	\addcontentsline{toc}{subsection}{}
	\subsection*{【提交记录】}
	
	https://www.luogu.com.cn/record/87885905。
	
	% -----------------------------------------------------------------------------------
	
	\section*{P5115 Check,Check,Check one two!}
	
	\phantomsection
	\addcontentsline{toc}{subsection}{}
	\subsection*{【题目描述】}
	
	https://www.luogu.com.cn/problem/P5115。
	
	现在给定一个长度为 $n\le 10^5$ 的字符串,希望您求出
	
	$$\sum_{1\leq i < j \leq n}lcp(i,j)lcs(i,j)[lcp(i,j)\leq k_1][lcs(i,j) \leq k_2]$$
	
	\phantomsection
	\addcontentsline{toc}{subsection}{}
	\subsection*{【题解】}
	
	首先先转化到两棵树上为树上 lca 相关的问题。
	
	一种比较勇猛的做法是,设一条边的权值为:当两端的深度都小于等于 $k$ 时,权值为边长;当一端深度小于等于 $k$ 时,边权为 $\min(dep_u,dep_v)$;否则边权为 $0$。
	
	对于每个 $1$ 树上的点 $u$,求出它子树内的点对在第二棵树上的贡献和,乘上 $(u,fa_u)$ 的边权贡献进答案。在第二棵树上可以用全局平衡二叉树快速维护加点删点求总贡献,第一棵树上进行一个配套的 dsu on tree 即可。$O(n\log^2n)$。
	
	实际上有 $26n$ 做法,大概是考虑 $[j-lcs,j+lcp]$ 和 $i-lcs,i+lcp$ 是一样的,所以不妨考虑某个后缀树上的结点 $T$ 的贡献。时代变了啊!
	
	\phantomsection
	\addcontentsline{toc}{subsection}{}
	\subsection*{【提交记录】}
	
	https://www.luogu.com.cn/record/87938469。
	
	% -----------------------------------------------------------------------------------
	
	\section*{【CF1060G】Balls and Pockets}
	
	\phantomsection
	\addcontentsline{toc}{subsection}{}
	\subsection*{【题目描述】}
	
	https://www.luogu.com.cn/problem/CF1060G。
	
	给出一个无限长的序列 $p_0,p_1,p_2,\ldots$,初始 $p_i=i$。
	
	给出 $n$ 个互不相同的整数 $a_1,a_2,\ldots,a_n$,可以对序列 $p$ 做以下操作若干次:将序列 $p$ 的第 $a_1,a_2,\ldots,a_n$ 项从序列 $p$ 中删掉,然后将剩余的数字按照原来在序列 $p$ 中的顺序重新排列作为新的序列 $p$。
	
	给出 $m$ 组询问,每组询问给出两个整数 $x,k$,询问进行 $k$ 次操作后的序列 $p$ 中 $p_x$ 的值。
	
	\phantomsection
	\addcontentsline{toc}{subsection}{}
	\subsection*{【题解】}
	
	这个变化过程非常复杂,我没有办法发现足够理想的性质,于是就得从比较生猛的维护去考虑。
	
	手玩足够的数据,并将每次删除的映射画出来看看,考虑这么求一个询问的答案,令 $f(x)=x+\sum_{i}[a_i-i+1\le x]$,那答案就是 $f^{(k)}(x)$。
	
	离线下来处理,从左到右维护每个 $x$ 右移的过程,对于 $x\in[a_i-i+1,a_i]$ 的 $x$ 都可以一起处理,因为它们的增量是比较统一的。分成两种情况考虑:
	
	\begin{itemize}
		\item 对于 $a_i\ge a_{i+1}-i$,那么 $[a_i-i+1,a_{i+1}-i]$ 间的询问都会加上 $i$,然后整体过度到 $a_{i+1}-i,a_{i+1}$ 的处理范畴内。
		\item 对于 $a_i<a_{i+1}-i$,一定可以分成两段,左边这段增加了 $(k+1)i$,右边这段增加了 $ki$,然后左边的这段跑到了右边的这段的右边。
	\end{itemize}

	我们拿一个平衡树维护这个过程,并维护子树剩余步数最小值,和坐标的整体平移;每当有剩余步数小于 $0$ 时,我们就在平衡树上找到哪个点,并计算它的答案。复杂度 $O(n\log_2n)$。
	
	\phantomsection
	\addcontentsline{toc}{subsection}{}
	\subsection*{【提交记录】}
	
	https://www.luogu.com.cn/record/88269582。
	
	% -----------------------------------------------------------------------------------
	
	\section*{P8056 C 图上的数}
	
	\phantomsection
	\addcontentsline{toc}{subsection}{}
	\subsection*{【题目描述】}
	
	https://www.luogu.com.cn/problem/P8056。
	
	给定一个 $n$ 个点 $m$ 条边的无向图(保证无重边无自环但不保证连通),每条边有一个 $1\sim m$ 的互不相同的编号。
	
	定义一条边是孤边,当且仅当它的两端点均已经被删除。
	
	您需要给定一个删点顺序,令 $P_i$ 表示第 $i$ 条变成孤边的边的编号,您需要最小化 $P_i$ 的字典序。
	
	若某一时刻存在多条边变为孤边,我们认为,编号小的边先变为孤边。$n,m\le 10^6$。
	
	\phantomsection
	\addcontentsline{toc}{subsection}{}
	\subsection*{【题解】}
	
	将边划分为 $0\sim 2$ 度三种。考虑贪心,每次的比较优的选择有两种:
	
	\begin{itemize}
		\item [1.] 选择一个点删掉,这个点对应的 $1$ 度的边的编号要被最小化。
		\item [2.] 选择一条 $2$ 度的边,删去它对应的两个点。
	\end{itemize}

	有了这个大方向,我们来考虑一些细节问题:
	
	\begin{itemize}
		\item 每次我们应该在两个操作中选一个比较小的进行。
		\item 若现在决定进行的 $2$ 操作的两个端点都没有相邻的 $1$ 度边,则可以以任意顺序删去这两点;存在一个端点有相邻的 $1$ 度边,则先把那个空着的给删了,然后把另一个删了;若两个端点都有相邻的 $1$ 度边,那么忽略这条边,因为两端点的决策都会被第一种贪心策略涵盖。
		\item 在某次删点操作中,它会删去所有和它相邻的 $1$ 度边;但如果有这么一条 $2$ 度边,它的另一个端点没有任何一条相邻的 $1$ 度边,并且它插入后有利于缩小字典序,那么我们也顺便把这条边删掉,也把那个点删掉。
	\end{itemize}

	讲真我觉得这个正确性好感性啊,但是又想不出很好的证明。$O(n\log_2n)$。
	
	\phantomsection
	\addcontentsline{toc}{subsection}{}
	\subsection*{【提交记录】}
	
	https://www.luogu.com.cn/record/88386505。
	
	% -----------------------------------------------------------------------------------
	
	\section*{【UR 19】通用测评号}
	
	\phantomsection
	\addcontentsline{toc}{subsection}{}
	\subsection*{【题目描述】}
	
	https://uoj.ac/problem/514。
	
	有 $n$ 个容量为 $a$ 的燃料箱,每次随机选择一个没有满的燃料箱,将其内的燃料量加一,直到每个燃料箱的燃料量都大于等于 $b$ 为止,问最后满的燃料箱的数量的期望。$n,a,b\le 250$。
	
	\phantomsection
	\addcontentsline{toc}{subsection}{}
	\subsection*{【题解】}
	
	显然存在一个燃料箱,它会被填充恰好 $b$ 次,且最后一次填充的是它;不妨假设这就是燃料箱 $1$,它的作用是 $b-1$ 次操作会被排列进其它燃料箱的填充序列里;而其它燃料箱彼此之间相对独立。
	
	一个错误的想法是,用二元生成函数,多加一维 $y$ 表示满的燃料箱,将 $y$ 带入 $1+y$,$na^2$ 多项式快速幂即可,最后的 $0$ 次项是总数,$1$ 次项的贡献和,贡献和除以总数得出期望。
	
	但是事实上,每种方案的概率不是均等的!只能通过概率乘以贡献的方式来算期望。
	
	注意到一开始的概率都是 $\frac 1{n}$,随后每填满一个燃料箱,概率就变成 $\frac 1{n-1},\frac 1{n-2},\cdots$。我们进行一个 dp,设 $f_{i,j}$ 表示前 $i$ 个操作,填满了 $j$ 个燃料箱,的对应方案,要乘上的概率的大小的和:
	
	$$
	\begin{aligned}
		&f_{i,j}\cdot \frac 1{n-j} \rightarrow f_{i+1,j} \\
		&f_{i,j}\cdot \frac 1{n-j}\cdot (n-j-1)\cdot \binom{i-ja}{b-1} f_{i+1,j+1}
	\end{aligned}
	$$
	
	而其它没有满的燃料箱的贡献,可以通过 EGF 累乘来计算。复杂度 $O(n^2a\log(na))$。
	
	\phantomsection
	\addcontentsline{toc}{subsection}{}
	\subsection*{【提交记录】}
	
	https://uoj.ac/submission/585682。
	
	% -----------------------------------------------------------------------------------
	
	\section*{【UR 17】滑稽树前做游戏}
	
	\phantomsection
	\addcontentsline{toc}{subsection}{}
	\subsection*{【题目描述】}
	
	https://uoj.ac/problem/372。
	
	$x_1,x_2,...,x_n$ 为 $n$ 个独立随机变量,且均服从 $[0,1]$ 上的均匀分布。给出 $m$ 对关系 $(a_i,b_i)$,令 $f(x)=\max\left(\max_{i=1}^n x_i,\max^m_{i=1}\left(x_{a_i}+x_{b_i}\right)\right)$,求 $f(x)$ 的期望。
	
	$n\le 30$,对 $998244353$ 取模。
	
	\phantomsection
	\addcontentsline{toc}{subsection}{}
	\subsection*{【题解】}
	
	首先,期望转为 $2-\int dx \cdot $ 小于等于 $x$ 的概率。设 $f(G,y,t)$ 表示,图 $G$ 上,每个点的权值都要小于等于 $y$,且任意两点的和都小于等于 $t$ 的概率。此处要求 $y\in[\frac t2,t]$。
	
	分类讨论:若所有的点的权值都小于 $\frac t2$,那么一定都合法;若存在点的权值大于等于 $\frac t2$,则我们枚举其最大值取的点 $u$,那么对于任意一个和 $u$ 相邻的点 $v$,$x_v\le t-x_u$;可以发现这些点 $v$ 被删除后,答案不变。设这张图被删去 $u$ 和 $\{v\}$ 后分裂成了连通子图 $G_{u,1\sim k}$,有:
	
	$$
	f_{G,y,t}=\left(\frac t2\right)^{|G|}+\sum_{u\in G}{\int_{x=\frac t2}^y dx\cdot (t-x)^{|\{v\}|}\cdot\prod f\left(G_{u,i},x,t\right)}
	$$
	
	注意到 $f$ 是一个关于 $y$ 和 $t$ 的 $|G|$ 次齐次多项式,并且这个 $t$ 可以当作常数,只有 $y$ 一直参与积分运算。我们把这个 dp 递归地进行下去并记忆化(不需要考虑图同构),那么这个玩意的复杂度就相当正确!
	
	最后只要对 $f(G,\min(t,1),t)dt$ 积分即可。
	
	\phantomsection
	\addcontentsline{toc}{subsection}{}
	\subsection*{【提交记录】}
	
	https://uoj.ac/submission/585272。
	
	% -----------------------------------------------------------------------------------
	
	\section*{【UR 17】滑稽树上滑稽果}
	
	\phantomsection
	\addcontentsline{toc}{subsection}{}
	\subsection*{【题目描述】}
	
	https://uoj.ac/problem/370。
	
	他有 $n$ 个滑稽果,第 $i$ 个滑稽果的大小为 $a_i$。
	
	他现在想把它们构成一棵任意形态的有根树,每个点的滑稽度为它的大小和它父亲的滑稽度的 $\operatorname{and}$。特别地,根的滑稽度等于他的大小。
	
	为了世界的和平,他希望能最小化这棵树上所有滑稽果的滑稽度之和。$n\le 10^5,a_i\le 2\times 10^5$。
	
	\phantomsection
	\addcontentsline{toc}{subsection}{}
	\subsection*{【题解】}
	
	注意到字典序最小的结论是错误的(否则也不会出现在部分分里)。注意到每个数的贡献最多在 $w$ 轮后就会收敛到所有数的交集。设 $f_S$ 表示 $S$ 前 $i$ 个数的 $\operatorname{and}$ 为 $S$ 的最小代价。那么可以 $3^w$ dp,也即 $a_i^{\log_23}$。
	
	\phantomsection
	\addcontentsline{toc}{subsection}{}
	\subsection*{【提交记录】}
	
	https://uoj.ac/submission/585117。
	
	% -----------------------------------------------------------------------------------
	
	\section*{【2020-2021 Summer Petrozavodsk Camp, Day 6: Korean Contest】Oneness}
	
	\phantomsection
	\addcontentsline{toc}{subsection}{}
	\subsection*{【题目描述】}
	
	https://codeforces.com/gym/102155/problem/E。
	
	给定一个随机产生的高精度数 $x\le 10^{2.5\times 10^5}$,求:
	
	$$\sum_{i=1}\left\lfloor\frac x{1111...1(i\ 1s)}\right\rfloor$$
	
	\phantomsection
	\addcontentsline{toc}{subsection}{}
	\subsection*{【题解】}
	
	首先求出:
	
	$$\sum_{i=1}\frac x{1111...1(i\ 1s)}$$
	
	以调和级数的复杂度求出 $\sum_{i=1}\frac 1{1111...1(i\ 1s)}$,然后 NTT 即可。然后要减去:
	
	$$
	\sum_{i=1}\left\{\frac x{1111...1(i\ 1s)}\right\}
	$$
	
	注意到:
	
	$$
	\left\{\frac x{1111...1(i\ 1s)}\right\}=\left\{\sum_{j=1} \left\{\frac 9{10^{ij}}\cdot x\right\}\right\}
	$$
	
	那么我们只要处理出所有 $\left\{\frac x{10^k}\right\}$ 即可。最后,对于过于接近整数的中间结果,调用一次高精度除法即可;由于数据随机,所以最多调用 $O(1)$ 次除法。
	
	
	\phantomsection
	\addcontentsline{toc}{subsection}{}
	\subsection*{【提交记录】}
	
	狗屎出题人卡我 $10^{-11}$ 级别的精度,狗屎。
	
	% -----------------------------------------------------------------------------------
	
	\section*{【2020-2021 Summer Petrozavodsk Camp, Day 6: Korean Contest】Koosaga's Problem}
	
	存疑!存疑!
	
	本场的 https://codeforces.com/gym/102155/problem/D 和 https://codeforces.com/gym/102984/problem/F 保留作模拟赛题目(看到这里的你,不要点进去哦)。
	
\end{document}


















\documentclass[UTF8,12pt,a4paper]{ctexart}
\usepackage[utf8]{inputenc}
\usepackage{color}
\usepackage{array}
\usepackage{diagbox}
\usepackage{multicol}
\usepackage{multirow}
\usepackage{CJKfntef}
\usepackage{booktabs}
\usepackage{fancyhdr}
\usepackage{graphicx}
\usepackage{lastpage}
\usepackage{indentfirst}
\usepackage{amsmath,amssymb}
\usepackage{tabu}
\usepackage[colorlinks,linkcolor=blue,anchorcolor=blue,citecolor=green,bookmarks=true]{hyperref}
%\usepackage[breaklinks,colorlinks,linkcolor=black,citecolor=black,urlcolor=black]{hyperref}
\usepackage{bookmark}
\usepackage{inconsolata}
\usepackage{geometry}
\usepackage{titlesec}
\usepackage{paralist}
\usepackage[ruled,linesnumbered]{Algorithm2e}
\setlength{\parindent}{2em}
\geometry{left=2.45cm,right=2.45cm,top=2.75cm,bottom=2.6cm}
\setcounter{secnumdepth}{0}
\let\itemize\compactitem
\let\enditemize\endcompactitem
\let\enumerate\compactenum
\let\endenumerate\endcompactenum
\let\description\compactdesc
\let\enddescription\endcompactdesc

\usepackage{booktabs} % for much better looking tables
\usepackage{array} % for better arrays (eg matrices) in maths
\usepackage{paralist} % very flexible & customisable lists (eg. enumerate/itemize, etc.)
\usepackage{verbatim} % adds environment for commenting out blocks of text & for better verbatim
\usepackage{subfig} % make it possible to include more than one captioned figure/table in a single float
\usepackage{xcolor}
\usepackage{endnotes}
\usepackage{multirow}
\usepackage{notoccite}
\usepackage{longtable}
\usepackage{geometry}
\usepackage{multicol}
\usepackage{multirow}
\usepackage{tabu}
\usepackage{xeCJK}
\usepackage{CJK}                   
\usepackage{CJKfntef}        
\usepackage{fancyhdr}               
\usepackage{graphicx}                 
\usepackage{lastpage}    
\usepackage{listings}
\usepackage{fancybox}
\usepackage{xcolor}
\usepackage{fontspec}
\usepackage{layout}
\usepackage{titletoc}
\usepackage{listings}
\usepackage{color}
\usepackage{xcolor}
\usepackage{ctex}
\usepackage{mathrsfs}
\usepackage{hyperref}
\lstset{
	basicstyle={      
		\color{black}
		\fontspec{Consolas}
	},
	keywordstyle={
		\color{blue}
		\fontspec{Consolas}
	},
	numberstyle={
		\color{black}
		\textbf
	},
	rulecolor=\color{blue},
	numbers=left,                               
	frame=single,                            
	frameround=tttt,
	morekeywords={Sample, Input, Output},   % 可以手动添加关键字
}
\setmonofont{Consolas}
\newcommand{\stress}[1]{\textbf{\CJKunderdot{#1}}}

\definecolor{mys}{rgb}{1,0.2,0}%一个颜色
\definecolor{codegray}{rgb}{0.5,0.5,0.5}
\lstset{
	numberstyle=\ttfamily\color{codegray}
}

\linespread{1.5}%修改行距

% \titleformat{\subsection}{\fontsize{13}{16}}

% \begin{lstlisting}
	% \end{lstlisting}

%%%my packages
\usepackage{hyperref}% Links
\hypersetup{
	colorlinks=true,
	linkcolor=mys,
	urlcolor=blue
}
% \usepackage{pseudo}% Pseudo Code

\title{标题(无实际意义)}
\author{作者(无实际意义)}
\date{日期时间(无实际意义)}

\begin{document}
	\fontsize{12pt}{12pt}\selectfont
	
	\newpage
	\pagestyle{fancy}
	\lhead{\footnotesize \songti{信息学一周解题报告汇总\ \ \ -LSH}}
	\cfoot{\footnotesize 第 \thepage \ 页\qquad  共  \pageref{LastPage} 页}
	
	\phantomsection
	\addcontentsline{toc}{section}{}
	\rhead{\footnotesize \songti{第五、六周\ \ \ 2022.10.9 $\sim$ 2022.10.22}}
	
	
	\section*{【CERC2017】Buffalo Barricades}
	
	\phantomsection
	\addcontentsline{toc}{subsection}{}
	\subsection*{【题目描述】}
	
	https://www.luogu.com.cn/problem/P4737。
	
	第一象限上有一些整点表示牛。接下来依次来了一些人,每个人放下一个树桩,然后从这个树桩向上向左延伸地建栅栏,直到碰到其它栅栏为止。问每个树桩在刚建成的时候它左下方向包住了多少只牛。牛和人的数量分别小于等于 $3\times 10^5$。
	
	\phantomsection
	\addcontentsline{toc}{subsection}{}
	\subsection*{【题解】}
	
	一种办法是,我们用一个平衡树维护每个区域内的牛和未来要放下的树桩,每次放下新树桩时,找到其对应的区域,并进行启发式分裂。平衡树可以按照 $x$ 排序,先把横坐标太大不可能被包住的部分 split 出去,然后在剩下的部分中,每次去掉一个最大值和一个最小值,直到剩下的点全部被包住或全不被包住为止,这样做时间复杂度为 $O(\min(a,b))$,总复杂度也就是 $O(n\log^2 n)$。
	
	另一种办法是,从上到下扫描线,维护当前还存在竖线,那么每次加入新树桩都属把它左侧时间比其晚的一个连续后缀去掉;最后变成一个合并问题就行。
	
	\phantomsection
	\addcontentsline{toc}{subsection}{}
	\subsection*{【提交记录】}
	
	https://www.luogu.com.cn/record/89790717。
	
	% -----------------------------------------------------------------------------------
	
	\section*{【CF1693D】Decinc Dividing}
	
	\phantomsection
	\addcontentsline{toc}{subsection}{}
	\subsection*{【题目描述】}
	
	https://www.luogu.com.cn/problem/CF1693D。
	
	定义一个序列 $a$ 是好的,仅当可以通过删除 $a$ 的一个单调递减子序列(可以为空)使得 $a$ 单调递增。  
	给定一个 $1\sim n$ 的排列 $p$,你需要求出 $p$ 有多少子段是好的。$n\le 2\times 10^5$。
	
	\phantomsection
	\addcontentsline{toc}{subsection}{}
	\subsection*{【题解】}
	
	注意到如果不存在 3 4 1 2 或者 2 1 4 3 这样的子序列,就一定合法。构造的话,直接提取出区间的 LIS 即可。
	
	对于每个 3,求出其后面最近的 4;对于每个 2,求出其前面最近的 1,然后从右到左扫描线,扫到一个 1 就把对应其的 2 全加进去,遇到一个 4 就把它对应的 3 全拿出来在树状数组上查询一个后缀最大值,就能找到所有的 3 4 1 2。
	
	找到所有 ban 掉的区间 $[l,r]$,然后双指针找到每个 $l$ 能延伸到的最大的 $r$ 即可。$O(n\log^2n)$。
	
	\phantomsection
	\addcontentsline{toc}{subsection}{}
	\subsection*{【提交记录】}
	
	https://www.luogu.com.cn/record/90355912。
	
	% -----------------------------------------------------------------------------------
	
	\section*{【P7482】不条理狂诗曲}
	
	\phantomsection
	\addcontentsline{toc}{subsection}{}
	\subsection*{【题目描述】}
	
	https://www.luogu.com.cn/problem/P7482。
	
	YSGH 有一个长度为 $n$ 的非负整数序列 $a$,定义 $f(l, r)$ 表示从 $a$ 序列的区间 $[l, r]$ 选择若干不相邻的数的和的最大值。$n\le 2\times 10^5$。
	
	YSGH 想知道 $\displaystyle \left[ \sum_{l = 1}^{n} \sum_{r = l}^{n} f(l, r) \right] \bmod ({10}^9 + 7)$ 。
	
	\phantomsection
	\addcontentsline{toc}{subsection}{}
	\subsection*{【题解】}
	
	考虑分治,对于分治中线的左右两侧,求出取了 mid 和没有取 mid 的两种后缀最大值,记为 $(f,g)$,然后两侧分别按照 $f-g$ 排序,然后双指针一下即可。
	
	\phantomsection
	\addcontentsline{toc}{subsection}{}
	\subsection*{【提交记录】}
	
	https://www.luogu.com.cn/record/90460076。
	
	% -----------------------------------------------------------------------------------
	
	\section*{「C.E.L.U-02」苦涩}
	
	\phantomsection
	\addcontentsline{toc}{subsection}{}
	\subsection*{【题目描述】}
	
	https://www.luogu.com.cn/problem/P7476。
	
	YQH 的脑中可以被分成 $n$ 个片区,每个片区相当于一个存放记忆的可重集,初始为空。他将进行 $m$ 次这三种操作:
	
	\begin{itemize}
		\item [1.] 区间 $l\sim r$ 的片区中都浮现了一个苦涩值为 $k$ 的记忆。
		\item [2.] YQH 开始清理 $l\sim r$ 片区的记忆。如果一个片区 $k\in[l,r]$ 且 $k$ 中苦涩值最大的记忆与 $l\sim r$ 片区中苦涩值最大的记忆相等,则将这个苦涩值最大的记忆忘记。如果在同一个片区有多个相同的苦涩值最大的记忆,则只忘记一个。如果这些片区内没有记忆,则无视。
		\item [3.] YQH 想知道,$l\sim r$ 片区中苦涩值最大的记忆的苦涩值是多少,如果不存在,输出`-1`。
	\end{itemize}

	$n\le 2\times 10^5$。
	
	\phantomsection
	\addcontentsline{toc}{subsection}{}
	\subsection*{【题解】}
	
	用线段树套堆来维护,每个结点维护一个最大值,再开一个堆维护当前还没下传的堆。
	
	遇到一个点,如果它的最大 tag 等于目标区间的最大值,那就将它下传;如果当前区间被目标区间完全包含,若它的最大 tag 恰好等于目标最大值,那就 pop;否则,暴力递归下去删并剪枝。
	
	根据势能分析,时间复杂度为 $O(n\log^2n)$。
	
	\phantomsection
	\addcontentsline{toc}{subsection}{}
	\subsection*{【提交记录】}
	
	https://www.luogu.com.cn/record/90466798。
	
	% -----------------------------------------------------------------------------------
	
	\section*{【JSOI2016】无界单词}
	
	\phantomsection
	\addcontentsline{toc}{subsection}{}
	\subsection*{【题目描述】}
	
	https://www.luogu.com.cn/problem/P5770。
	
	对于一个单词 $S$ ,如果存在一个长度 $l$,满足 $0 < l < |S|$,并且使得 $S$ 长度为 $l$ 的前缀与 $S$ 长度为 $l$ 的后缀相同,JYY 则称 $S$ 是有界的。比如 `aabaa` 和 `ababab` 就都是有界的字符串。如果一个单词不存在这样的 $l$ ,则 JYY 称之为无界单词。
	
	现在考虑所有仅由字母 `a` 和 `b` 组成的长度为 $N(N\le 64)$ 的字符串,JYY 想知道:
	
	\begin{itemize}
		\item [1.] 一共有多少个无界单词?
		\item [2.] 这些无界单词中,按字典序排列第 $K$ 小的单词是哪一个?
	\end{itemize}
	
	\phantomsection
	\addcontentsline{toc}{subsection}{}
	\subsection*{【题解】}
	
	二分,变成求字典序小于等于 $mid$ 的无界单词有多少个;枚举在哪一位开始严格小于,然后用那个经典的枚举最短 border(最短 border 小于等于一半)即可。$O(64^5)$。
	
	事实上,我们可以从高到低位逐位确定,就直接变成了固定某个前缀,任意后缀,问无界单词的个数;而 dp 所需要的信息可以通过 kmp 和 z 函数求出,总复杂度可以降至 $O(64^3)$。
	
	\phantomsection
	\addcontentsline{toc}{subsection}{}
	\subsection*{【提交记录】}
	
	https://www.luogu.com.cn/record/90599280。
	
	% -----------------------------------------------------------------------------------
	
	\section*{【HNOI2014】江南乐}
	
	\phantomsection
	\addcontentsline{toc}{subsection}{}
	\subsection*{【题目描述】}
	
	https://www.luogu.com.cn/problem/P3235。
	
	游戏的规则:给定一个数 $F$,游戏包含 $N$ 堆石子,小 $A$ 和他的对手轮流操作。每次操作时,操作者先选定一个不小于 $2$ 的正整数 $M$,指定一堆数量不小于 $F$ 的石子分成 $M$ 堆,并且满足这M堆石子中石子数最多的一堆至多比石子数最少的一堆多 $1$。当一个玩家不能操作的时候,他就输掉。问谁会获胜?每堆石子的数量小于等于 $10^5$。
	
	\phantomsection
	\addcontentsline{toc}{subsection}{【CERC2017】Buffalo Barricades}
	\subsection*{【题解】}
	
	考虑一堆石头会转移到哪些状态,注意到 $M\le 10^{2.5}$ 时,总的方案数当然在 $10^{2.5}$ 级别;当 $M> 10^{2.5}$ 时,每堆石头的总数也都小于等于根号,可以直接枚举 $x=\left\lfloor\frac NM\right\rfloor$,然后判断 $N-xy\left(y\in\left[\frac Nx, \frac{N}{\lfloor\frac Nx\rfloor}\right]\right)$ 的奇偶性。
	
	总复杂度 $O(V^{1.5})$。
	
	\phantomsection
	\addcontentsline{toc}{subsection}{}
	\subsection*{【提交记录】}
	
	https://www.luogu.com.cn/record/90607712。
	
	% -----------------------------------------------------------------------------------
	
	\section*{【JLOI2014】镜面通道}
	
	\phantomsection
	\addcontentsline{toc}{subsection}{}
	\subsection*{【题目描述】}
	
	https://www.luogu.com.cn/problem/P3260。
	
	在一个二维平面上,有一个镜面通道,由镜面 $AC, BD$ 组成,$AC, BD$ 长度相等,且都平行于 $x$ 轴,$B$ 位于 $(0,0)$。
	
	通道中有 $n$ 个外表面为镜面的光学元件,光学元件 $\alpha$ 为圆形,光学元件 $beta$ 为矩形(这些元件可以与其他元件和通道有交集,具体看下图)。光线可以在 $AB$ 上任一点以任意角度射入通道,光线不会发生削弱。当出现元件与元件,元件和通道刚好接触的情况视为光线无法透过(比如两圆相切)。
	
	现在给出通道中所有元件的信息($\alpha$ 元件包括圆心坐标和半径 $x_i, y_i, r_i$,$\beta$ 元件包括左下角和右上角坐标 $x_1, y_1, x_2, y_2$),请求出至少拿走多少个光学元件后,存在一条光线线路可以从 $CD$ 射出。
	
	\phantomsection
	\addcontentsline{toc}{subsection}{}
	\subsection*{【题解】}
	
	首先猜想只要左侧和右侧连通就一定能找到光路。那么我们对每个原件建一个点,记底部为 $S$,顶部为 $T$,在有相交或相切的两个元件之间连边,那么问题就转化为了点最小割。
	
	把每个点拆成两个点(之间为容量 $1$ 的边),每条边拆成两条边(容量 $+\infty$)即可。
	
	\phantomsection
	\addcontentsline{toc}{subsection}{}
	\subsection*{【提交记录】}
	
	https://www.luogu.com.cn/record/90628002。
	
	% -----------------------------------------------------------------------------------
	
	\section*{【SDOI2016】平凡的骰子}
	
	\phantomsection
	\addcontentsline{toc}{subsection}{}
	\subsection*{【题目描述】}
	
	https://www.luogu.com.cn/problem/P4080。
	
	这是一枚平凡的骰子。它是一个均质凸多面体,表面有 $n$ 个端点,有 $f$ 个面,每一面是一个凸多边形,且任意两面不共面。将这枚骰子抛向空中,骰子落地的时候不会发生二次弹跳(这是一种非常理想的情况)。
	
	你希望知道最终每一面着地的概率。
	
	\phantomsection
	\addcontentsline{toc}{subsection}{}
	\subsection*{【题解】}
	
	首先我们如何求出重心的位置?对于多边形,我们可以三角剖分,三角形的重心就是三点的平均点,然后加权平均得到总的重心;对于多面体,我们可以四棱锥剖分,四棱锥的重心就是四点的平均点,总质量等比于体积可以用行列式算出来,然后求加权平均。
	
	然后对于每一面,三角剖分成三角形,然后用题面中的公式算出每个三角形对应的球表面积,求出总的那一面对应的球表面积。对于公式中二面角的求值,可以找两个向量外积得出一个面的法向量,然后用内积求出两个法向量之间的夹角。
	
	复杂度线性于输入。
	
	\phantomsection
	\addcontentsline{toc}{subsection}{}
	\subsection*{【提交记录】}
	
	https://www.luogu.com.cn/record/90962381。
	
	% -----------------------------------------------------------------------------------
	
	\section*{【SDOI2016】平凡的骰子}
	
	\phantomsection
	\addcontentsline{toc}{subsection}{}
	\subsection*{【题目描述】}
	
	https://www.luogu.com.cn/problem/P4080。
	
	这是一枚平凡的骰子。它是一个均质凸多面体,表面有 $n$ 个端点,有 $f$ 个面,每一面是一个凸多边形,且任意两面不共面。将这枚骰子抛向空中,骰子落地的时候不会发生二次弹跳(这是一种非常理想的情况)。
	
	你希望知道最终每一面着地的概率。
	
	\phantomsection
	\addcontentsline{toc}{subsection}{}
	\subsection*{【题解】}
	
	首先我们如何求出重心的位置?对于多边形,我们可以三角剖分,三角形的重心就是三点的平均点,然后加权平均得到总的重心;对于多面体,我们可以四棱锥剖分,四棱锥的重心就是四点的平均点,总质量等比于体积可以用行列式算出来,然后求加权平均。
	
	然后对于每一面,三角剖分成三角形,然后用题面中的公式算出每个三角形对应的球表面积,求出总的那一面对应的球表面积。对于公式中二面角的求值,可以找两个向量外积得出一个面的法向量,然后用内积求出两个法向量之间的夹角。
	
	复杂度线性于输入。
	
	\phantomsection
	\addcontentsline{toc}{subsection}{}
	\subsection*{【提交记录】}
	
	https://www.luogu.com.cn/record/90962381。
	
	% -----------------------------------------------------------------------------------
	
	\section*{【JOISC 2021 Day3】ビーバーの会合 2}
	
	\phantomsection
	\addcontentsline{toc}{subsection}{}
	\subsection*{【题目描述】}
	
	https://www.luogu.com.cn/problem/P7565。
	
	给定一棵有 $N$ 个点的树,每一个点上有一个人,这些人要开秘密会议。
	
	假设一次秘密会议有 $P$ 个人参加,这 $P$ 个人分别在第 $p_1,p_2,\cdots,p_P$ 个点上。如果点 $k$ 满足下面这个值最小($d(a,b)$ 为点 $a$ 到点 $b$ 的距离,$k$ 不需要满足 $k \in \{p_1,p_2,\cdots,p_P\}$):
	
	$$\sum\limits_{i=1}^Pd(k,p_i)$$
	
	那么就称第 $k$ 个点为可期待的,这场会议的期待值即为所有点中中可期待点的个数。
	
	对于每个 $j \in [1,N]$,求当会议里有 $j$ 个人的时候,会议的期待值的最大值是多少。$N\le 2\times 10^5$。
	
	\phantomsection
	\addcontentsline{toc}{subsection}{}
	\subsection*{【题解】}
	
	首先相当于找到一条极长的链,满足其两侧子树大小的 $\min\times 2$ 能达到参会人数。我们在启发式合并的过程中维护,考虑重<轻、重>轻、轻<轻、重<祖先、重>祖先、轻<祖先、轻>祖先等情况,结合离线建线段树处理重儿子的方法就可以 $O(n\log^2n)$。
	
	实际上我们只要按照 $\min(sz,n-sz)$ 从小到大加入边,维护每个连通块的直径(合并直径可以枚举六种搭配)即可 $O(n\log^2n)$。
	
	\phantomsection
	\addcontentsline{toc}{subsection}{}
	\subsection*{【提交记录】}
	
	https://www.luogu.com.cn/record/89241848。
	
	% -----------------------------------------------------------------------------------
	
	\section*{【CERC2013】Captain Obvious and the Rabbit-Man}
	
	\phantomsection
	\addcontentsline{toc}{subsection}{}
	\subsection*{【题目描述】}
	
	https://www.luogu.com.cn/problem/P7016。
	
	定义 $p_i=\sum\limits_{j=1}^ka_j\times F_j^i$。现在给定 $k,m$ 以及 $\{p_i\}_{i=1}^k$,请求出 $p_{k+1}\bmod m$。($F$ 为斐波那契数列)。
	
	\phantomsection
	\addcontentsline{toc}{subsection}{}
	\subsection*{【题解】}
	
	这玩意显然是要你用 $p_{1\sim k}$ 线性表出 $p_{k+1}$,有一种很神仙的构造是另 $A(x)=\prod_{i=1}^n(x-F_i)$,然后把 $1,2,3,5,\cdots$ 带进去并求和。
	
	比较能想的做法是,这个玩意是一个范德蒙德矩阵,而范德蒙德矩阵的逆矩阵有个式子,可以通过 dp 得出。
	
	\phantomsection
	\addcontentsline{toc}{subsection}{}
	\subsection*{【提交记录】}
	
	https://www.luogu.com.cn/record/89748940。
	
	% -----------------------------------------------------------------------------------
	
	\section*{【CF898E】Mod Mod Mod}
	
	\phantomsection
	\addcontentsline{toc}{subsection}{}
	\subsection*{【题目描述】}
	
	https://www.luogu.com.cn/problem/CF889E。
	
	$$f(x,n) = x \mod a_n$$
	
	$$f(x,i) = ( x \mod a_i ) + f(x \mod a_i,i+1)$$
	
	给出 $a$ 序列,当 $x$ 取遍所有非负整数时 $f(x,1)$ 的最大值。$n\le 2\times 10^5, a_i\le 10^{13}$。
	
	\phantomsection
	\addcontentsline{toc}{subsection}{}
	\subsection*{【题解】}
	
	注意到一个 $x$ 最多被操作 $log_2(a)$ 次。而这个 $x$ 一定会使某个 $x_i$ 恰好顶到 $a_i-1$。
	
	如果枚举这个 $a_i-1$ 是在哪里顶到的,那后半部分的贡献相当好计算,但是前半部分的贡献我们发现很难计算,甚至可能需要一个 dp。
	
	所以我们还是考虑正着做,考虑一个 $x$ 在前 $i$ 个的贡献和为 $g(x,i)$,另 $x\leftarrow x-1$,则 $g(x,i)\leftarrow g(x,i)-i+???$。我们不妨把 $???$ 放缩掉,因为当它会产生 $???$ 时,也就对应了那个 $x_i$ 恰好为 $a_i-1$。
	
	于是我们就可以定于一个 dp 数组 $h$,记 $h_{i,R}$ 表示 $x_i=R$ 时,$g(x,i)-ik$ 这个一次函数的最大一次项是多少。只做这两种转移:1. 对于 $R>a_{i+1}$,转移向 $R\bmod a_{i+1}$;2. 对于 $R=a_{i+1}-1$,从大于等于它的第一个值转移来;而对于 $a_{i+1}<R$,其一次项显然不变。
	
	这个 dp 数组可以涵盖所有需要的 dp 值,也就依赖于上两段中的那个性质。$O(n\log^2(a))$。
	
	\phantomsection
	\addcontentsline{toc}{subsection}{}
	\subsection*{【提交记录】}
	
	https://www.luogu.com.cn/record/89943431。
	
	% -----------------------------------------------------------------------------------
	
	\section*{Nim 积}
	
	\phantomsection
	\addcontentsline{toc}{subsection}{}
	\subsection*{【题目描述】}
	
	https://loj.ac/p/179。
	
	对于两个非负整数 $x, y$ 我们定义其 Nim 积 $x\otimes y$:
	
	$$
	x \otimes y = \operatorname {mex} \{ (a\otimes b) \oplus (a\otimes y) \oplus (x\otimes b) \mid 0\le a < x \wedge 0\le b < y  \}
	$$
	
	生成的数据均在 $2^{32}$ 范围以内,故保证 $0\le x, y < 2^{32}$。四组数据中的 $T$ 分别为 $10, 1000, 3\times 10^4, 3\times 10^7$。
	
	\phantomsection
	\addcontentsline{toc}{subsection}{}
	\subsection*{【题解】}
	
	有以下几个关键性质:
	
	$$
	2^{2^k}\otimes x=2^{2^k}\cdot x (x<2^{2^k})\\
	2^{2^k}\otimes 2^{2^k} = \frac 32 2^{2^k}\\
	$$
	
	根据分配律,将 $x\otimes y$ 拆成 $\log^2$ 个 $2^{x’}\otimes 2^{y'}$,再将 $2^{x’}\otimes 2^{y'}$ 拆成:$\left(\prod 2^{2^{x_i}}\right)\otimes \left(\prod 2^{2^{y_i}}\right)$。
	
	考虑若 $x'$ 和 $y'$ 的最高位相同,那么答案 $\otimes $ 上 $\frac 23 2^{2^{\mathrm{highestBit}}}$;否则,答案 $\otimes $ 上 $2^{2^{\mathrm{highestBit}}}$。递归下去计算即可。
	
	为了加速,我们可以预处理出所有 $x$ 中的 $4$ 位和 $y$ 中的 $4$ 位的 Nim 积,即可做到单次 $O(16)$。
	
	\phantomsection
	\addcontentsline{toc}{subsection}{}
	\subsection*{【提交记录】}
	
	https://loj.ac/s/1608964。
	
	% -----------------------------------------------------------------------------------
	
	\section*{【北京省选集训2019】图的难题}
	
	\phantomsection
	\addcontentsline{toc}{subsection}{}
	\subsection*{【题目描述】}
	
	https://www.luogu.com.cn/problem/P5295。
	
	书上画出了一张无向图,要求把边染成黑白两色,要求所有白色边构成的子图没有环,且所有黑色边构成的子图没有环。  
	小 D 无论怎样尝试都觉得书上的问题没有解,她想请你帮她确认一下。
	
	\phantomsection
	\addcontentsline{toc}{subsection}{}
	\subsection*{【题解】}
	
	我们大胆猜想如果每张导出子图都满足边数小于等于 $2\times$ 点数 $-2$,那么就一定有解。问题就变为了一个最大非空闭合子图问题。
	
	可以强制某个边满流并多次跑最大流来完成;也可以通过退流,以不增加的时间复杂度完成此问题。
	
	\phantomsection
	\addcontentsline{toc}{subsection}{}
	\subsection*{【提交记录】}
	
	https://www.luogu.com.cn/record/90388457。
	
	% -----------------------------------------------------------------------------------
	
	\section*{「CCO 2018 Day2」Flop Sorting}
	
	\phantomsection
	\addcontentsline{toc}{subsection}{}
	\subsection*{【题目描述】}
	
	https://loj.ac/p/3519。
	
	给定一个 $1$ 到 $N$ 的排列 $a_i$,我们规定一次「翻牌」操作表示交换一个区间的最小值与最大值的位置。现在给定你 $Q$ 次翻牌操作,每次对 $[l,r]$ 执行翻牌操作,求进行 $Q$ 次翻牌操作后的最终序列。
	
	现在给定了 $N$,初始序列 $a_i$ 和最终序列,求中间要进行的翻牌操作。$1 \le a_i \le N \le 4096$,$1 \le Q \le 3 \times 10^5$。
	
	\phantomsection
	\addcontentsline{toc}{subsection}{}
	\subsection*{【题解】}
	
	写了个乱搞做法,摁艹,艹不过去啊。
	
	正解考虑归并排序,但是这个归并相当难归并,我们考虑再用一个类似归并形状的东西来处理归并——把左区间的一个后缀翻转,右区间的一个前缀翻转,使得中间它们正好接上;然后把它们一起翻转,然后就变成了左边的两个有序数列和右边的两个有序数列的归并。
	
	据说时间复杂度是 $O(n\log n)$,但我不会证,而且莫名其妙地拿了最优解。
	
	\phantomsection
	\addcontentsline{toc}{subsection}{}
	\subsection*{【提交记录】}
	
	https://loj.ac/s/1609113。
	
	
\end{document}


















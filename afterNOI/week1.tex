\documentclass[UTF8,12pt,a4paper]{ctexart}
\usepackage[utf8]{inputenc}
\usepackage{color}
\usepackage{array}
\usepackage{diagbox}
\usepackage{multicol}
\usepackage{multirow}
\usepackage{CJKfntef}
\usepackage{booktabs}
\usepackage{fancyhdr}
\usepackage{graphicx}
\usepackage{lastpage}
\usepackage{indentfirst}
\usepackage{amsmath,amssymb}
\usepackage{tabu}
\usepackage[colorlinks,linkcolor=blue,anchorcolor=blue,citecolor=green,bookmarks=true]{hyperref}
%\usepackage[breaklinks,colorlinks,linkcolor=black,citecolor=black,urlcolor=black]{hyperref}
\usepackage{bookmark}
\usepackage{inconsolata}
\usepackage{geometry}
\usepackage{titlesec}
\usepackage{paralist}
\setlength{\parindent}{2em}
\geometry{left=2.45cm,right=2.45cm,top=2.75cm,bottom=2.6cm}
\setcounter{secnumdepth}{0}
\let\itemize\compactitem
\let\enditemize\endcompactitem
\let\enumerate\compactenum
\let\endenumerate\endcompactenum
\let\description\compactdesc
\let\enddescription\endcompactdesc

\usepackage{booktabs} % for much better looking tables
\usepackage{array} % for better arrays (eg matrices) in maths
\usepackage{paralist} % very flexible & customisable lists (eg. enumerate/itemize, etc.)
\usepackage{verbatim} % adds environment for commenting out blocks of text & for better verbatim
\usepackage{subfig} % make it possible to include more than one captioned figure/table in a single float
\usepackage{xcolor}
\usepackage{endnotes}
\usepackage{multirow}
\usepackage{notoccite}
\usepackage{longtable}
\usepackage{geometry}
\usepackage{multicol}
\usepackage{multirow}
\usepackage{tabu}
\usepackage{xeCJK}
\usepackage{CJK}                   
\usepackage{CJKfntef}        
\usepackage{fancyhdr}               
\usepackage{graphicx}                 
\usepackage{lastpage}    
\usepackage{listings}
\usepackage{fancybox}
\usepackage{xcolor}
\usepackage{fontspec}
\usepackage{layout}
\usepackage{titletoc}
\usepackage{listings}
\usepackage{color}
\usepackage{xcolor}
\usepackage{ctex}
\usepackage{mathrsfs}
\usepackage{hyperref}
\lstset{
	basicstyle={      
		\color{black}
		\fontspec{Consolas}
	},
	keywordstyle={
		\color{blue}
		\fontspec{Consolas}
	},
	numberstyle={
		\color{black}
		\textbf
	},
	rulecolor=\color{blue},
	numbers=left,                               
	frame=single,                            
	frameround=tttt,
	morekeywords={Sample, Input, Output},   % 可以手动添加关键字
}
\setmonofont{Consolas}
\newcommand{\stress}[1]{\textbf{\CJKunderdot{#1}}}

\definecolor{mys}{rgb}{1,0.2,0}%一个颜色
\definecolor{codegray}{rgb}{0.5,0.5,0.5}
\lstset{
	numberstyle=\ttfamily\color{codegray}
}

\linespread{1.5}%修改行距

% \titleformat{\subsection}{\fontsize{13}{16}}

% \begin{lstlisting}
	% \end{lstlisting}

%%%my packages
\usepackage{hyperref}% Links
\hypersetup{
	colorlinks=true,
	linkcolor=mys,
	urlcolor=blue
}
% \usepackage{pseudo}% Pseudo Code

\title{标题(无实际意义)}
\author{作者(无实际意义)}
\date{日期时间(无实际意义)}

\begin{document}
	\fontsize{12pt}{12pt}\selectfont
	
	\newpage
	\pagestyle{fancy}
	\lhead{\footnotesize \songti{信息学一周解题报告汇总\ \ \ -LSH}}
	\cfoot{\footnotesize 第 \thepage \ 页\qquad  共  \pageref{LastPage} 页}
	
	\phantomsection
	\addcontentsline{toc}{section}{}
	\rhead{\footnotesize \songti{第一周\ \ \ 2022.9.5 $\sim$ 2022.9.11}}
	
	
	\section*{【北大集训2021】经典游戏}
	
	\phantomsection
	\addcontentsline{toc}{subsection}{}
	\subsection*{【题目描述】}
	
	https://uoj.ac/problem/719。
	
	Alice 和 Bob 在一棵有根树上博弈,树的每个点 $i$ 上有 $a_i$ 个棋子,两个人轮流选择一个非叶子节点上的棋子,将其移入其子树的一个节点上(这个节点不能是自己),并且 Bob 可以在游戏开始前往某一个节点上放一个棋子。需要支持两种操作:单点修改 $a_i$、查询点 $u$ 的邻域中有多少个点为根时 $Alice$ 必胜。
	
	$n,q\le 10^6$。
	
	
	\phantomsection
	\addcontentsline{toc}{subsection}{}
	\subsection*{【题解】}
	
	首先,每个棋子是独立的,所以可以将所有每个棋子的 SG 函数异或起来得到总的异或值;而单个棋子的 SG 值就是其子树中离它最远的叶子节点。
	
	这棵树暂时没有根,我们不妨给其找一个根;“离 $u$ 最远”的限制使得这个根最适合放在\textbf{直径的中点}上(如果直径上点数为偶数,那么可以有两个根),可以推出一个性质:当“根”为 $u$ 的祖先时,$SG(u)$ 为其子树内的离其最远点;当根为 $u$ 或者位于 $u$ 的子树内时,$SG(u)$ 为 $u$ 到另一个直径端点的距离。记下这两个值分别为 $up_u$ 和 $down_u$。
	
	Alice 必胜,则意味着 $\oplus SG_u> up_{root}$。考虑快速统计 $root$ 有多少个孩子满足如上条件,可以发现其所有孩子 $up$ 都等于 $up_u+1$,所以将所有孩子的 SG 值插进 Trie 里即可快速查询;而修改可以被转换为两个 dfn 序上的区间异或。拿一个树状数组维护异或的 tag 即可。
	
	
	\phantomsection
	\addcontentsline{toc}{subsection}{}
	\subsection*{【提交记录】}
	
	https://uoj.ac/submission/581557。
	
	% -----------------------------------------------------------------------------------
	
	\section*{【WC2014】时空穿梭}
	
	\phantomsection
	\addcontentsline{toc}{subsection}{}
	\subsection*{【题目描述】}
	
	https://www.luogu.com.cn/problem/P4152。
	
	小X驾驶着他的飞船准备穿梭过一个 $n$ 维空间,这个空间里每个点的坐标可以用 $n$ 个实数来表示,即 $(x_1, x_2, ... , x_n)$ 。为了穿过这个空间,小 X 需要在这个空间中选取 $c$ $(c \geq 2)$ 个点作为飞船停留的地方,而这些点需要满足以下三个条件:
	
	\begin{itemize}
		\item[1.] 每个点的每一维坐标均为正整数,且第 $i$ 维坐标不超过 $m_i$ 。
		\item[2.] 第 $i + 1$ $(1 \leq i < c)$ 个点的第 $j$ $(1 \leq j \leq n)$ 维坐标必须严格大于第 $i$ 个点的第 $j$ 维坐标。
		\item [3.] 存在一条直线经过所选的所有点。在这个 $n$ 维空间里,一条直线可以用 $2n$个实数 $p_1$, $p_2$, … , $p_n$, $v_1$, $v_2$, … , $v_n$ 表示。 直线经过点 $(x_1, x_2, ... , x_n)$ ,当且仅当存在实数 $t$,使得对 $i = 1$ … $n$ 均满足 $x_i$ = $p_i + tv_i$。
	\end{itemize}
	
	问方案数 $\pmod{998244353}$。$T\le 100,n\le 11,c\le 20,m_i\le 10^5$。	
	
	\phantomsection
	\addcontentsline{toc}{subsection}{}
	\subsection*{【题解】}
	
	我们枚举相邻两点间距离的最简比序列,可以发现填数的方案数只和总长有关,所以只需要枚举序列的总和,插板法计算序列的具体方案数,然后把“最简比”莫比乌斯反演掉。求方案数时,每维独立,对于每一维分别枚举其放大的倍数并将方案数求和(是一个等差数列求和),最后将每一维相乘即可。
	
	预处理组合数,预处理 $c=2\sim 20$ 的所有需要的数组,借助快速莫比乌斯反演可以将复杂度降低到勉强能过的程度。
	
	
	\phantomsection
	\addcontentsline{toc}{subsection}{}
	\subsection*{【提交记录】}
	
	https://www.luogu.com.cn/record/85925954。
	
	% -----------------------------------------------------------------------------------
	
	\section*{【POI2012】PRE-Prefixuffix}
	
	\phantomsection
	\addcontentsline{toc}{subsection}{}
	\subsection*{【题目描述】}
	
	https://www.luogu.com.cn/problem/P3546。
	
	对于两个串 $S_1$、$S_2$,如果能够将 $S_1$ 的一个后缀移动到开头后变成 $S_2$,就称 $S_1$ 和 $S_2$ 循环相同。例如串 ababba 和串 abbaab 是循环相同的。
	
	给出一个长度为 $n$ 的串 $S$,求满足下面条件的最大的 $L$:$S$ 的 $L$ 前缀和 $S$ 的 $L$ 后缀是循环相同的。
	
	$n\le 10^6$。
	
	\phantomsection
	\addcontentsline{toc}{subsection}{}
	\subsection*{【题解】}
	
	枚举 $S$ 的 Border,记当前 Border 的长度为 $i$,那么我们就是要求 $S[i+1,n-i]$ 的最长 Border。记 $T$ 的最长 Border $f(T)$ 考虑到 $f(S[i+2,n-i-1])\ge f(S[i+1,n-i])-2$,所以我们 $i$ 从大到小计算,只需要 check $\le f(S[i+2,n-i-1]])+2$ 的即可。从大到小 check,不难发现均摊的复杂度是对的。
	
	另一种理解是:将字符串变为 $s_1,s_n,s_2,s_{n-1},s_3,\cdots$,那么我们要找其一个回文前缀 $i$,和 $S[i+1,2n]$ 的最长回文前缀长度相加即可。两种方法是等价的,因为方法一其实本质上也是 manacher。
	
	
	\phantomsection
	\addcontentsline{toc}{subsection}{}
	\subsection*{【提交记录】}
	
	https://www.luogu.com.cn/record/85928701。
	
	% -----------------------------------------------------------------------------------
	
	\section*{【CF1178H】Stock Exchange}
	
	\phantomsection
	\addcontentsline{toc}{subsection}{}
	\subsection*{【题目描述】}
	
	https://www.luogu.com.cn/problem/CF1178H。
	
	股票交易所里有 $2n$ 种股票,每种股票有两个属性 $a_i,b_i$,在时刻 $t\ge 0$,第 $i$ 种股票的价格为 $a_i*\lfloor t\rfloor+b_i$。
	
	每个时刻可以进行任意次股票交易,在时刻 $t$ 时能够把股票 $i$ 换成股票 $j$ 当且仅当股票 $i$ 在时刻 $t$ 的价格不小于股票 $j$ 在时刻 $t$ 的价格。
	
	现在你手上有 $1$ 到 $n$ 号股票各一张,现在要求的是把这些股票换成 $n+1$ 到 $2n$ 号股票各一张的最早时刻,以及在最早换完股票前提下的最少交易次数。
	
	$1\le n\le 2200,0\le a_i,b_i\le 10^9$。
	
	\phantomsection
	\addcontentsline{toc}{subsection}{}
	\subsection*{【题解】}
	
	由一次函数的性质,得存在一种最优方案,使得只有第 $0$ 天和第 $T$ 天有交换股票。我们考虑二分 + 最大流流判断,然后用费用流求最小代价。
	
	二分 check 部分的优化:在第 $0$ 天将所有股票换成其能换成的 $a_iT+b_i$ 最大的股票,最后贪心 check。
	
	费用流部分的优化:每个点直接以 $1$ 的代价连向第 $0$ 天排序结果的一个前缀,以 $2$ 的代价连向它能连向的第 $T$ 天排序结果的一个前缀,即可将边数将至 $O(n)$。
	
	
	\phantomsection
	\addcontentsline{toc}{subsection}{}
	\subsection*{【提交记录】}
	
	https://www.luogu.com.cn/record/85963059。
	
	% -----------------------------------------------------------------------------------
	
	\section*{【IOI2019】天桥}
	
	\phantomsection
	\addcontentsline{toc}{subsection}{}
	\subsection*{【题目描述】}
	
	https://www.luogu.com.cn/problem/P5812。
	
	有 $n$ 个高度分别为 $h_{1\sim n}$ 的大楼分布在一条数轴上,有 $m$ 座天桥从 $(x_1,y)$ 横亘至 $(x_2,y)$,穿过了与其相交的大楼。问 $s$ 大楼底部到 $t$ 大楼底部的最短路(只能走大楼和天桥)。
	
	$n,m\le 10^5$。
	
	\phantomsection
	\addcontentsline{toc}{subsection}{}
	\subsection*{【题解】}
	
	这张图的问题在于点数太多。记关键点为如下两类:
	
	\begin{itemize}
		\item[1.] 天桥的左右两端点。
		\item[2.] $s$ 或者 $t$ 到某一个天桥的前驱和后继。
	\end{itemize}

	我们只保留关键点及关键点下方第一个位于天桥上的点,并跑最短路即可。$O(n\log_2(n))$。
	
	
	\phantomsection
	\addcontentsline{toc}{subsection}{}
	\subsection*{【提交记录】}
	
	https://www.luogu.com.cn/record/85969190。
	
	% -----------------------------------------------------------------------------------
	
	\section*{Koishi Loves Construction}
	
	\phantomsection
	\addcontentsline{toc}{subsection}{}
	\subsection*{【题目描述】}
	
	https://www.luogu.com.cn/problem/P3599。
	
	Koishi 决定走出幻想乡成为数学大师!
	
	Flandre 听说她数学学的很好,就给 Koishi 出了这样一道构造题:
	
	Task1:试判断能否构造并构造一个长度为 $n$ 的 $1 \dots n$ 的排列,满足其 $n$ 个前缀和在模 $n$ 的意义下互不相同。
	
	Task2:试判断能否构造并构造一个长度为 $n$ 的 $1 \dots n$ 的排列,满足其 $n$ 个前缀积在模 $n$ 的意义下互不相同。
	
	\phantomsection
	\addcontentsline{toc}{subsection}{}
	\subsection*{【题解】}
	
	对于第一问,若 $n\neq 1$ 且 $n$ 为奇数,则由等差数列求和的式子可以看出无解;否则我们这样构造:$0,-1,2,-3,4,\cdots$(其实也就是一个 Zig-zag?)。
	
	对于第二问,若 $n\neq 4$ 且 $n$ 不是素数,则因为会提早出现 $0$ 而导致无解。否则,求出原根并用上面的构造即可(最后一位放 $0$)。
	
	
	\phantomsection
	\addcontentsline{toc}{subsection}{}
	\subsection*{【提交记录】}
	
	https://www.luogu.com.cn/record/85771488。
	
	% -----------------------------------------------------------------------------------
	
	\section*{【CF1523H】 Hopping Around the Array}
	
	\phantomsection
	\addcontentsline{toc}{subsection}{}
	\subsection*{【题目描述】}
	
	https://www.luogu.com.cn/problem/CF1523H。
	
	给定一个序列 $a_1,a_2,\dots,a_n$,在第 $i$ 个点可以一步走到 $[i,i+a_i]$ 中的任何一个点,多次询问,每一次从 $l$ 出发,你可以删掉不超过 $k$ 个点,将剩下的点重标号,使得到达 $r$ 的步数最少。
	
	$n\leq 2\times 10^4$,$k\leq 30$。
	
	\phantomsection
	\addcontentsline{toc}{subsection}{}
	\subsection*{【题解】}
	
	题目的要求是$i$ 可以走向 $[i,i+a_i]$,这个 $i+a_i$ 是非常关键的变量,我们一定是从 $i+a_i$ 小的走向 $i+a_i$ 大的;并且 $i$ 可以走向 $i$,所以 $t$ 秒钟能 $i$ 走到的位置一定构成一个区间 $[i,j]$。
	
	采取类似倍增的思路,设 $f_{i,j,k}$ 表示从 $i$ 出发走 $2^j$ 秒用了 $k$ 个删除,能走到的区间长度。那么转移时,需要维护一个 $f$ 的 ST 表,预处理 ST 表的复杂度过大。
	
	注意到,区间 $[i,i+f_{i,j-1,k_1})$ 里最优的转移点,一定是 $i+a_i$ 最大的点,我们只需要预处理出 $i+a_i$ 序列的 ST 表即可。	
	
	\phantomsection
	\addcontentsline{toc}{subsection}{}
	\subsection*{【提交记录】}
	
	https://www.luogu.com.cn/record/85766661。
	
	% -----------------------------------------------------------------------------------
	
	\section*{赵云八卦阵}
	
	\phantomsection
	\addcontentsline{toc}{subsection}{}
	\subsection*{【题目描述】}
	
	https://uoj.ac/problem/703。
	
	给定一个序列 $a_1,\cdots,a_n$,每次可以将某一项 $a_x$($2\le x\le n$)变成 $a_x\oplus a_{x−1}$,求这个序列最长上升子序列长度的最大值。
	
	$n\le 10^6,1\le a_i<2^{60}$。
	
	\phantomsection
	\addcontentsline{toc}{subsection}{}
	\subsection*{【题解】}
	
	首先,$a_i$ 之间相互独立,且每个 $a_i$ 可以成为 $a_i\oplus v$,其中 $v$ 可以用 $a_{1\sim i-1}$ 线性表出。
	
	注意到 $a_{1\sim i}$ 构成的线性基会至多分成 $60$ 段,我们找到这些段,称段首为关键点。关键点的取值集合,被它后面这一段的取值集合包含,所以如果要将关键点算入答案,那么后面这一段的所有非关键点都必然被算入答案(否则可以调整)。进而得知选中的非关键点一定是一个后缀。
	
	从后向前 dp,记 $f_{i,j}$ 表示后 $i$ 段选择了 $j$ 个关键点时,最左边这个数的最大值,那么借助在线性基上查排名、根据排名查数、查前驱,即可转移。
	
	线性基上排名和值互相转化:求出线性基的最小表示,将在基中的位提出来单独看,则有排名 = 值,于是便可以互相转化了。
	
	线性基上查前驱:即对于 $X$,求出最大的 $X\oplus v$,使得 $v\in B$ 且$X\oplus v < B$。从高到低位扫一遍,若 $X\oplus B_i<X$ 则 $X\leftarrow X\oplus B_i$;再从高到低位扫一遍,若 $X\oplus B_i<Y$,则 $X\leftarrow X\oplus B_i$。
	
	\phantomsection
	\addcontentsline{toc}{subsection}{}
	\subsection*{【提交记录】}
	
	https://uoj.ac/submission/581859(最优解!)。
	
	% -----------------------------------------------------------------------------------
	
	\section*{【UR23】地铁规划}
	
	\phantomsection
	\addcontentsline{toc}{subsection}{}
	\subsection*{【题目描述】}
	
	https://uoj.ac/problem/693。
	
	有一个长度为 $m$ 的边序列,你需要使用交互库提供的可撤销并查集接口,在线地对于每个 $l$ 求出最大的 $r$,使得区间 $[l,r]$ 内的边不形成环。
	
	$m\le 2\times 10^5$。
	
	\phantomsection
	\addcontentsline{toc}{subsection}{}
	\subsection*{【题解】}
	
	Baka's Trick:即谢队法,可以支持 $l$ 和 $r$ 分别递增的优化建图、不具有可减性的双指针等等,需要用两个栈/前缀和式数据结构,当左栈全部弹出后,再将右栈全部倒出来插入左栈,如此往复推进。
	
	单栈模拟队列:用一个栈维护上述的双栈,栈内每个元素分别标记为 0 或者 1,其中 0 表示左栈元素,0 元素构成的子序列应自栈顶向栈底递增,1 元素应当自栈顶向栈底递减。每次顺利右移 $l$ 需要保证栈顶为 $0$ 元素,记栈内中 0 元素的个数为 cnt,那么不断地弹出栈顶,直到弹出 $\operatorname{lowbit}(cnt)$ 个 0 元素为止,然后先把 1 元素插回去,再把 0 元素插回去;$cnt=0$ 时,我们重构整个栈使得全部变为 0 元素。
	
	可以发现总复杂度是对的:对于 0 元素:$\sum_{i=1}^k \operatorname{lowbit}(i)=\sum_{i=0}^{\log_2(k)} 2^i \left\lfloor \frac{n}{2^i} \right\rfloor - \left\lfloor \frac{n}{2^{i + 1}} \right\rfloor=O(n\log_2(n))$;对于 $1$ 元素:也一样可以观察到只会被遍历 $O(\log_2(n))$ 次。
	
	\phantomsection
	\addcontentsline{toc}{subsection}{}
	\subsection*{【提交记录】}
	
	https://uoj.ac/submission/581960。
	
	% -----------------------------------------------------------------------------------
	
	\section*{【ZJOI2022】众数}
	
	\phantomsection
	\addcontentsline{toc}{subsection}{}
	\subsection*{【题目描述】}
	
	https://uoj.ac/problem/741。
	
	可怜得到了一个序列 $a_1,a_2,\cdots,a_n$, 她可以对这个序列使用一次超能力:选择一个区间 $[l,r](1\le l\le r\le n)$ 和一个整数 $k\in [−10^9,10^9]$,将区间内的所有数 $a_l,a_{l+1},\cdots,a_r$ 加上 $k$。
	
	九条可怜很喜欢长得比较一致的序列,因此她希望最终的序列众数的出现次数尽可能多。给出序列 $a$,你需要输出最终序列的众数出现次数的最大值,并输出这个众数的所有可能取值。注意对于一个序列,众数的取值可能不止一个。
	
	$\sum n\le 5\times 10^5$。
	
	\phantomsection
	\addcontentsline{toc}{subsection}{}
	\subsection*{【题解】}
	
	对于每个数分别算出其最多出现的次数。
	
	根号分治,对于出现次数大于根号的数配上另一个数:把序列中其它的值分别遍历一遍,求出其前缀和的差的最大值,求两次,分别用来更新这两者的答案。
	
	对于出现次数小于根号的数配上出现次数小于根号的数:对于每个 $l$,预处理出 $f_{l,k}$ 表示最小的 $r$ 使得 $[l,r]$ 内的众数出现次数达到 $k$,那么 $k$ 只需要到 $\sqrt{n}$ 就好。我们用一个桶套桶维护双指针,离线下所有的询问区间众数的 $[l,r]$ 的双指针,扫上 $\sqrt{n}$ 遍就好了。
	
	\phantomsection
	\addcontentsline{toc}{subsection}{}
	\subsection*{【提交记录】}
	
	https://uoj.ac/submission/581619。
	
	% -----------------------------------------------------------------------------------
	
	\section*{【BalticOI 2022 Day1】 Uplifting Excursion}
	
	\phantomsection
	\addcontentsline{toc}{subsection}{}
	\subsection*{【题目描述】}
	
	https://www.luogu.com.cn/problem/P8392。
	
	有 $2m+1$ 种物品,重量分别为 $-m,-m+1,\ldots, m-1,m$。重量为 $i$ 的物品有 $a_i$ 个。
	
	你需要拿走若干物品,使得这些物品重量之和恰好为 $l$。在此基础上,你需要拿尽可能多的物品。
	
	问在物品重量之和恰好为 $l$ 的基础上,你最多能拿多少物品。
	
	$1\leq m \leq 300$,$-10^{18}\le l \le 10^{18}$,$0\le a_i\le 10^{12}$。
	
	\phantomsection
	\addcontentsline{toc}{subsection}{}
	\subsection*{【题解】}
	
	我们先选上所有数,然后贪心地删值最大数使得重量小于等于 $l$(或者反过来),然后试着去调整,发现调整的过程中一定不会使得值超过 $m^2$,所以只要维护一个大小为 $m^2$ 的背包即可。
	
	\phantomsection
	\addcontentsline{toc}{subsection}{}
	\subsection*{【提交记录】}
	
	https://www.luogu.com.cn/record/86035628。
	
	% -----------------------------------------------------------------------------------
	
	\section*{「Wdsr-2.7」八云蓝自动机 $Ⅱ$}
	
	\phantomsection
	\addcontentsline{toc}{subsection}{}
	\subsection*{【题目描述】}
	
	https://www.luogu.com.cn/problem/P7709。
	
	八云蓝自动机维护了一个长度为 $n$ 的序列 $A$ ,每个元素都有一个初始值。同时自动机会支持以下三种操作:
	
	\begin{itemize}
		\item ${\verb!1 l r k!}$ :将区间 $[l,r]$ 内的所有数字全都变为 $k$ ,即 $A_l\gets k,A_{l+1}\gets k,\cdots ,A_r\gets k$ 。
		\item ${\verb!2 x y!}$ :交换 $A_x$ 与 $A_y$ 的值。
		\item ${\verb!3 x!}$ :查询 $A_x$ 的值。
	\end{itemize}
	
	为了测试八云蓝自动机的效率,紫需要进行非常非常多次的测试。为了生成每个测试的所有操作,紫构造出了一个长度为 $m$ 的操作序列 $B$ , $B$ 中的元素就是八云蓝自动机可以执行的一个操作。
	
	设 $\Upsilon(l,r)$ 表示从初始状态开始,依次执行 $B_l,B_{l+1},\cdots B_r$ 操作后,所有操作 $3$ 的结果之和。特别地,如果这些操作中没有操作 $3$ ,那么 $\Upsilon(l,r)=0$ 。
	
	紫会向八云蓝自动机发起 $q$ 次询问,每次给出一组 $(l,r,p)$ ,八云蓝自动机需要计算出 
	
	$$\left(\sum_{i=l}^r \Upsilon(i,p)\right) \mod 2^{32}$$
	
	$1 \le n,m,q \le 3 \times 10 ^ 5$,$1 \le a_i,k \le 10^9;1 \le op \le 3;1 \le x,y \le n;x \neq y$。
	
	\phantomsection
	\addcontentsline{toc}{subsection}{}
	\subsection*{【题解】}
	
	我们考虑如何求出一个 $Ask(x)$ 的值:找到它前面第一个涉及到它的修改,若其位置已经小于 $l$,那么直接返回 $A_x$;否则,若其为区间修改操作,则直接返回这个区间覆盖的值;若为交换操作,则返回递归返回 $Ask(y)$ 的值。我们把所有的区间覆盖当作根节点,交换操作视作两个 $Ask()$ 操作,把每个 $Ask$ 连向它依赖的操作,形成了一个操作树。
	
	在操作树上求 $Ask(x)$ 的方法是,找到这个 $Ask()$ 操作对应的节点,然后找到它上方的最浅的在给定区间内的节点 $grand$,然后对于那个点返回 $A_x$ 或者返回区间覆盖的值。
	
	我们考虑从小到大扫描一遍,固定 $p$,对每个 $l$ 求出 $\sum_{i=l}^p \Upsilon(i,p)$。把这个拆成每个 $grand$ 的贡献,贡献的形式一定是对区间 $(time\left(fat_{grand}\right),time(grand)]$ 内分别贡献 $A_x$。
	
	考虑根号重构,重构的部分用差分前缀和即可解决;散块的贡献,我们离线下来,用 $O(\sqrt{N}) - O(1)$ 分块即可解决。复杂度 $O\left(n^{1.5}\right)$。
	
	\phantomsection
	\addcontentsline{toc}{subsection}{}
	\subsection*{【提交记录】}
	
	https://www.luogu.com.cn/record/86045324。
	
	% -----------------------------------------------------------------------------------
	
	\section*{【CF859G】Circle of Numbers}
	
	\phantomsection
	\addcontentsline{toc}{subsection}{}
	\subsection*{【题目描述】}
	
	https://www.luogu.com.cn/problem/CF859G。
	
	给定一个 $n$ 次单位根构成的式子,问它是否等于 $0$。$n\le 10^5$。
	
	\phantomsection
	\addcontentsline{toc}{subsection}{}
	\subsection*{【题解】}
	
	考虑其物理意义:等于 $0$ 当且仅当其重心位于 $O$ 点。直接用 double 当然是行不通的,我们多取几个质数,利用其单位根来判一下是否为 $0$ 即可。
	
	更为简洁的做法是考虑分圆多项式的性质,令 $a$ 循环卷积卷上 $\prod_{p|n}\left(1-x^p\right)$,若等于 $0$ 则原式为 $0$。$O(nw)$。
	
	\phantomsection
	\addcontentsline{toc}{subsection}{}
	\subsection*{【提交记录】}
	
	https://www.luogu.com.cn/record/86056257。
	
	% -----------------------------------------------------------------------------------
	
	\section*{【CF1344E】Train Tracks}
	
	\phantomsection
	\addcontentsline{toc}{subsection}{}
	\subsection*{【题目描述】}
	
	https://www.luogu.com.cn/problem/CF1344E。
	
	给定 $n$ 个点的一棵树,以 $1$ 为根,边有边权。
	
	有 $m$ 辆从 $1$ 开始到 $s_i$ 的火车,每个火车有一个初始时刻 $t_i$,在初始时刻时从根出发,向着目标 $s_i$ 前进,当火车在 $x$ 时刻到达一个点 $u$ 时,假设下一个路径上的点是 $v$,两点间边权是 $d$,则在 $x+d$ 时刻到达 $v$,火车在到达目标点后停止。
	
	由于每个点可能可以到达多个点,每个点有且仅有一个当前时刻可以到的儿子,每秒钟只能切换某一个点可以到的儿子,切换比火车开动先进行,若一辆火车走向了非目标方向的点,则立刻自爆。每个点初始能到的儿子是确定的。求出能否不发生自爆,如果不能,第一次自爆最晚什么时候发生?在此基础上,至少切换多少次一个点可以到的儿子。
	
	$n,m\le 10^5$。
	
	\phantomsection
	\addcontentsline{toc}{subsection}{}
	\subsection*{【题解】}
	
	我们可以把每辆火车拆成一些限制,然后用贪心来 check 这些性质能否同时被满足。但是限制有点多!
	
	我们注意到这题中每个点恰有一个出边的性质和 LCT 是一样的,所以结合 access 的复杂度分析,限制的总量是 $O(n\log_2(n))$ 的。
	
	\phantomsection
	\addcontentsline{toc}{subsection}{}
	\subsection*{【提交记录】}
	
	https://www.luogu.com.cn/record/86078364。
	
	% -----------------------------------------------------------------------------------
	
	\section*{【CF765G】Math, math everywhere}
	
	\phantomsection
	\addcontentsline{toc}{subsection}{}
	\subsection*{【题目描述】}
	
	https://www.luogu.com.cn/problem/CF765G。
	
	给定 $N$ 的因式分解形式 $\prod_{i=1}^np_i^{\alpha_i}$ 和一个长为 $m$ 的字符串 $s$。
	
	求 $k$ 的个数,使得 $0\leq k<N$ 且
	
	$$\forall i\in[0,m),\gcd(k+i,N)=1 \text{  if and only if  }s_i=1$$
	
	模 $10^9+7$。$m \leq 40$,$n\leq 5\times 10^5$,$p_i, \alpha_i \leq 10^9$。
	
	\phantomsection
	\addcontentsline{toc}{subsection}{}
	\subsection*{【题解】}
	
	首先我们发现 $s$ 只和 $k$ 模每个 $p_i$ 的余数有关,所以 $\alpha_i$ 是没有用的,所以直接把答案乘上 $p_i^{\alpha_i-1}$ 即可,然后每个素数其实就相当独立了,每个素数的作用是使得 $s_i$ 中的一些位变成 $0$,于是就有了一个 $2^{40}\cdot n$ 的暴力 dp。分为三段优化:
	
	\begin{itemize}
		\item $p\in [1,\frac k2)$:直接暴力用 HashMap 和队列维护这个 dp。
		\item $p\in [\frac k2, m)$:我们发现中间几位的具体状态是没用用的,我们直接记下中间几位中还有多少个数需要变成 $0$ 即可。
		\item $p\ge m$:我们直接记下一共有多少个数需要变成 $0$ 即可。
	\end{itemize}

	及时在 dp 中去除无用信息,以使得状态总量可控。
	
	\phantomsection
	\addcontentsline{toc}{subsection}{}
	\subsection*{【提交记录】}
	
	https://www.luogu.com.cn/record/86093959(在洛谷上仅次于 Bot)。
	
	% -----------------------------------------------------------------------------------
	
	\section*{【CF1710E】Two Arrays}
	
	\phantomsection
	\addcontentsline{toc}{subsection}{}
	\subsection*{【题目描述】}
	
	https://www.luogu.com.cn/problem/CF1710E。
	
	现有两个整数数组 $a_1, a_2, \dots, a_n$ 和 $b_1, b_2, \dots, b_m$。Alice 和 Bob 将要玩一个游戏,Alice 先手,然后他们轮流进行操作。他们在一个 $n\times m$ 的网格上进行游戏(网格有 $n$ 排 $m$ 列)。刚开始,有一个棋子放在网格的 $(1, 1)$ 上。在 Alice 或 Bob 轮次中,玩家可以选择以下两个动作中的一个进行操作:
	
	\begin{itemize}
		\item[1.] 将棋子移动到一个不同的格子上,该格子必须和棋子的原位置在同排或者同列上。玩家不能将棋子移动到已经被访问过 $1000$ 次的格子上。
		\item[2.] 以 $a_r + b_c$ 的得分立刻结束游戏,$(r, c)$ 表示棋子当前所在的单元格。
	\end{itemize}
	
	Bob 想要最大化自己的得分,Alice 则想要最小化自己的得分。如果他们都以最佳方式玩这个游戏,则最终的得分是多少?$n,m\le 10^3$。
	
	\phantomsection
	\addcontentsline{toc}{subsection}{}
	\subsection*{【题解】}
	
	我们可以二分一个 $mid$,使得 $a_r+b_c\le mid$ 的状态为 Alice 的必胜态,$a_r+b_c>mid$ 的状态为 Bob 的必胜态,双方都要避免走入对方的必胜态,这就变成了一个二分图博弈。
	
	注意到 $1000$ 次是没有什么作用的,因为只要来回消耗一下,就能消成 $1$。而二分图最大匹配很难求,考虑转化为最大独立集。
	
	我们枚举一个点 $(x,y)$,在其左上方全取 Alice 的必胜点,右下全取 Bob 的必胜点,假设我们能枚举所有 $(x,y)$ 就可以求出最大独立集。而事实上,$x$ 递增时最优的 $y$ 也递增,满足凸性,所以直接双指针维护一下即可。求出有 $(1,1)$ 和没有 $(1,1)$ 的最大独立集,判断其是否相等即可。
	
	\phantomsection
	\addcontentsline{toc}{subsection}{}
	\subsection*{【提交记录】}
	
	https://www.luogu.com.cn/record/86105128。
	
	% -----------------------------------------------------------------------------------
	
	\section*{【POI2006】PAL-Palindromes}
	
	\phantomsection
	\addcontentsline{toc}{subsection}{}
	\subsection*{【题目描述】}
	
	https://www.luogu.com.cn/problem/P3449。
	
	他写下了 $n$ 个回文串,随后将这些串两两组合,合并成一个新串。容易看出,一共会有 $n^2$ 个新串。
	
	两个串组合时顺序是任意的,即 `a` 和 `b` 可以组合成 `ab` 和 `ba`,另外自己和自己组合也是允许的。
	
	现在他想知道这些新串中有多少个回文串,你能帮帮他吗?$\sum len \le 2\times 10^6$。
	
	\phantomsection
	\addcontentsline{toc}{subsection}{}
	\subsection*{【题解】}
	
	两个回文串拼起来是回文串,当且仅当两个串的最小严格周期内的字符完全相等。
	
	\phantomsection
	\addcontentsline{toc}{subsection}{}
	\subsection*{【提交记录】}
	
	https://www.luogu.com.cn/record/86122697。
	
	% -----------------------------------------------------------------------------------
	
	\section*{【CF1693E】Outermost Maximums}
	
	\phantomsection
	\addcontentsline{toc}{subsection}{}
	\subsection*{【题目描述】}
	
	https://www.luogu.com.cn/problem/CF1693E。
	
	有一个长度为 $n+2$ 的序列 $a$,其中 $a_0=a_{n+1}=0$,其余元素均给定。  
	你可以进行下面两种操作任意次:
	
	1. 设 $x$ 表示序列 $a$ 最靠左的最大值的位置,则令 $a_x\leftarrow \max_{i=0}^{x-1}a_i$。
	2. 设 $y$ 表示序列 $a$ 最靠右的最大值的位置,则令 $a_y\leftarrow \max_{i=y+1}^{n+1}a_i$。
	
	你需要求出使序列 $a$ 的所有元素均变成 $0$ 所需的最少的操作总次数。$n\le 10^5$。
	
	\phantomsection
	\addcontentsline{toc}{subsection}{}
	\subsection*{【题解】}
	
	首先有一个 $O(n^2)$ 的贪心:每次把一个数变成其左边或者右边的最大值中比较小的一个。考虑优化。
	
	一种是从贡献的角度去优化:我们考虑一个数的贡献次数,在数轴上开一个线段树,记它左边的数为黑点,右边的数为白点,那么它每次是变成一个距离它最近的白点或者距离它最近的黑点,然后问走到 $0$ 的最优步数。我们直接用线段树维护区间矩阵连乘积,然后依次把黑点变成白点即可。
	
	另一种优化是尝试去强硬的优化其过程:贪心时我们不立刻确定那个最大值的方向,而是把这些最大值标记为“不确定”。再去考虑次大值,设次大值最左边在 $l$,最右边在 $r$,则会将所有 $>l$ 的“不确定”转化为“向右”,将所有 $<r$ 的“不确定”转化为“向左”;将所有 $<r$ 的“向右”计算上贡献并且改为不确定,再将所有 $>l$ 的向左计算上贡献并且改为“不确定”。于是我们发现“向左”都是一个前缀,“向右是一个后缀”,“不确定”是中间的一段。直接用树状数组维护区间和,再用变量 $[ql,qr]$ 维护“不确定”的两端点即可。
	
	\phantomsection
	\addcontentsline{toc}{subsection}{}
	\subsection*{【提交记录】}
	
	https://www.luogu.com.cn/record/86133353。
	
	% -----------------------------------------------------------------------------------
	
	\section*{色多项式}
	
	\phantomsection
	\addcontentsline{toc}{subsection}{}
	\subsection*{【题目描述】}
	
	https://loj.ac/p/6787。
	
	对于一个无向图 $G = (V, E)$,色多项式 $P(x)$ 是一个 $|V|$ 次多项式,对于任何正整数 $k$,$P(k)$ 为 $G$ 的顶点的 $k$ 染色的数量。
	
	给定一个图 $G$,请你计算出 $G$ 的色多项式系数。$n\le 21$。
	
	\phantomsection
	\addcontentsline{toc}{subsection}{}
	\subsection*{【题解】}
	
	一种是 $2^n\cdot n^3$ 集合幂级数做法,大力卡常!
	
	我的办法是,用广义串并联图缩点,对于每条边维护其两端点相同和两端点不同时的色多项式;然后爆搜,从小到大搜颜色,用一个桶维护多项式的和,然后再乘上组合数对应的多项式。实际上当然是可以卡掉的,但是出题人没卡,最优解!
	
	\phantomsection
	\addcontentsline{toc}{subsection}{}
	\subsection*{【提交记录】}
	
	https://loj.ac/s/1576105。
	
	
	
	
\end{document}














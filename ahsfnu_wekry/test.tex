\documentclass[12pt, a4paper, oneside]{ctexart}
\usepackage{amsmath, amsthm, amssymb, appendix, bm, graphicx, hyperref, mathrsfs, multirow}

\title{\textbf{My Wekry 7}}
\author{Linshey}
\date{\today}
\linespread{1.5}
\newtheorem{theorem}{定理}[subsection]
\newtheorem{definition}{定义}[subsection]
\renewcommand{\proofname}{\indent\bf 证明}



\begin{document}

\maketitle
\setcounter{page}{1}
\thispagestyle{empty}

\newpage
\pagenumbering{Roman}
\setcounter{page}{1}
\tableofcontents

\newpage
\setcounter{page}{1}
\pagenumbering{arabic}

\section{完成情况}

\begin{table}[h!]
\centering
\begin{tabular}{cccc}
    题目编号 & 题目链接 & 完成进度 & done\\\hline
    2 & \href{https://dmoj.ca/problem/cco23p2}{CCO '23 P2 - Real Mountains} & 听了做法 & \\
    3 & \href{https://dmoj.ca/problem/cco23p2}{CCO '23 P3 - Line Town
    } & 口胡 + 题解 & √ \\
    4 & \href{http://www.nfls.com.cn:10611/contest/889/problem/3}{高维碎块} & 题解 & √ \\
    5 & \href{https://www.luogu.com.cn/problem/P8257}{[CTS2022] 普罗霍洛夫卡} & 通过 + 题解 & √ \\
    6 & \href{https://loj.ac/p/3271}{「JOISC 2020 Day1」建筑装饰 4} & 口胡 & √ \\
    7 & \href{https://loj.ac/p/3272}{「JOISC 2020 Day1」汉堡肉} & 看了题解 & \\
    8 & \href{https://loj.ac/p/3276}{「JOISC 2020 Day2」遗迹} & 题解 & √ \\
    9 & \href{https://loj.ac/p/3277}{「JOISC 2020 Day3」星座 3} & 题解 & √ \\
    10 & \href{https://www.luogu.com.cn/problem/CF1810G}{The Maximum Prefix} & 题解 & √ \\
    11 & \href{https://www.luogu.com.cn/problem/CF1621G}{Weighted Increasing Subsequences} & 题解 & √ \\
    12 & \href{https://www.luogu.com.cn/problem/AT_agc062_d}{[AGC062D] Walk Around Neighborhood} & 题解 & √ \\
    13 & \href{https://www.luogu.com.cn/problem/P6072}{P6072 『MdOI R1』Path} & 口胡 + 题解 & √ \\
    14 & 海亮 Day1.B 计数 & 题解 & √ \\
    15 & 海亮 Day1.C 构造 & 题解 & √ \\
    16 & \href{https://qoj.ac/contest/1268/problem/6661}{IOI 2023 国家队集训@威海 Day 2 野餐} & 题解 & √ \\
    17 & \href{https://qoj.ac/contest/1268/problem/6659}{IOI 2023 国家队集训@威海 Day 2 外环路 2} & 题解 & √
\end{tabular}
\end{table}

\newpage
\section{解法记录}

\subsection{高维碎块}

\subsubsection{题意}

给定高维立方体 $[p_1,p_2,\cdots,p_n]$,若一个点 $x_{1\sim n}$ 可以走到 $y_{1\sim n}$ 当且仅当 $\forall i,y_i\ge x_i$ 且 $\sum y_i-x_i=1$,求这个 DAG 的最小链覆盖。

$n\le 32$,$p\le 10^9$,对质数取模。

\subsubsection{题解}

\begin{theorem}
    设 $M=\left\lfloor\frac {\sum_{i=1}^n p_i}2 \right\rfloor$,则最长反链 $I=\left\{\vec{x} \mid \sum_{i=1}^n {\vec{x}}_{i}=M\right\}$。
\end{theorem}

\begin{proof}
    因为所有 $\sum x_i$ 相同的点互不可达,所以 $|I|$ 至少为其中最大的一组的点数;又因为可以归纳构造出那么多条不相交链,则可以说明答案的小于等于定理中的 $I$。
\end{proof}

容斥后,题目转化为如下:

\begin{align*}
    \mathrm{Ans}=&\sum_{S} \binom{M-s(S)-1}{n-1}[M-s(S)\ge n]
\end{align*}

考虑折半:

\begin{align*}
    \mathrm{Ans}=&\sum_{S_0,S_1} \binom{M-s(S_0)-s(S_1)-1}{n-1}[M-s(S_0)-s(S_1)\ge n]\\
    =&\sum_{i=0}^{n-1}\sum_{S_0} \binom{M-s(S_0)-1}{i}\sum_{S_0,S_1} \binom{-s(S_1)}{n-1-i}[M-s(S_0)-s(S_1)\ge n]
\end{align*}

排序并双指针即可。\textbf{最长反链的题目,经常是需要你手动将点分层,然后证明其中一层,它恰好就是你的最长反链}。

\subsection{[CTS2022] 普罗霍洛夫卡}

首先将问题转化为:维护一个 $a_{1\sim n}$ 数组,每次操作将一个区间 $+1$,查询区间历史异或和。这个转化是相当大胆的,但是这能做下去的原因在于:保证 $a_{1\sim n}$ 不增,并且 $a_i-a_{i+1}\le 1$,这带来了相当多的额外性质,且待接下来使用。

历史异或和,不妨在每个 $a_i$ 发生改变时,把\textbf{旧的 $a_i$ 贡献到一个表示古早版本(非当前版本)历史异或和的数组 $H_i$ 上,再单独维护维护当前的 $a_i$ 对当前历史异或和的贡献}。这个分开考虑的合理性来源于,$a_i$ 的古早版本的历史异或和,它已经固定了;而 $a_i$ 的当前版本,随着时间的延申,对于当前的历史异或和的贡献是不同的,\textbf{这两者有本质差异}。

注意到 $a_i$ 当前版本延申了奇数次和偶数次\textbf{必须}分开来维护,我们维护 $f_i,g_i$ 表示,在奇数/偶数时刻产生的 $a_i$ 的异或和,再额外把 $H_i$ 考虑进去,得到 $f_i,g_i$ 的定义(记 $t_i$ 表示 $a_i$ 的产生时刻):

\begin{definition}
    \begin{align*}
        f_i=H_i\oplus (a_i\cdot[t_i \bmod 2 = 0])\\
        g_i=H_i\oplus (a_i\cdot[t_i \bmod 2 = 1])
    \end{align*}
\end{definition}

考虑一次修改的影响,化简后即为:$a_i\leftarrow a_i+1$,以及 $f_i\leftarrow f_i\oplus a_i \oplus (a_{i}+1)$ 或者 $g_i \leftarrow g_i \oplus a_i \oplus (a_{i}+1)$。\textbf{这意味着,不论 $t_i\bmod 2=0/1$,它对于 $f_i/g_i$ 有着相同的贡献形式!}$f_i$ 的定义中包括了 $H_i$ 的用意就在于此。

注意到 $DeltaF_i=a_i\oplus (a_i+1)=2^{\operatorname{lowbit}(a_i+1)+1}-1$,这个形式结合上 $a_i-a_{i+1}\le 1$,使得贡献 $2^j$ 的数小于等于 $\frac{n}{2^j}$ 种,这个贡献的形式比较好。

设 $B=7$,我们每 $2^B$ 个数分一块,\textbf{那么同一个块中最多只有一种 $a_i$ 使得 $DeltaF_i>2^B$!}我们每次快速找到这种数,就可以处理 $>2^B$ 的部分的异或和。我们要考虑的是怎样快速计算 $<2^B$ 的位对于整块 $f/g$ 异或和的贡献以及如何快速下传给每一个 $f_i,g_i$。

考虑一个块不断整体加一,整体加一第 $k$ 次,在 $2^i$ 位上的贡献有一个关于 $k$ 的\textbf{循环节} $2^i$!我们直接处理一个数组 $h_{i,k\in[0,2^i)}$ 表示每个 $k$ 在 $2^i$ 位上的贡献,进一步 $O(B)$ 算出 $g_{k\in[0,2^B)}$ 表示每一个时刻,它对于 $f/g$ 的异或和的贡献。于是就能 $O\left(2^B\right)$ pushup 使得接下来可以 $O(1)$ 整体加一;标记下传是这个过程的逆过程,无需赘述。

多么美好的题目!它无比巧妙地使用了子区间种类数异或和带来的漂亮的性质,完成了这样一个挺震撼的事情。究竟 lxl 是从做到哪一步时,才开始相信这样一个题目可做的呢?

\subsection{The Maximum Prefix}

这个 $h_{0\sim n}$,相当让人想要把 $f_{i,j}$ 表示前 $i$ 个数,最大前缀和为 $j$ 的概率求出来,从而使得对于 $h_i$ 的贡献可以分别计算。但是很可惜,这办不到 $O(n^2)$。

我们考虑弱于如上做法的处理方式,设 $f_{i,tar,0/1}$ 表示前 $i$ 个数、离目标前缀和最大值还有 $tar$ 的差距、是否达到过最大值的期望贡献和。\textbf{这个方式相当于是把不同的 $h$ 放进了同一个 dp 数组里一起转移}。

\subsection{Weighted Increasing Subsequences}

考虑每个数 $a_x$ 的贡献。\textbf{$a_x$ 没有贡献的充要条件是:设 $z$ 为 $\max\{i|a_i\ge a_x\}$,上升子序列的末尾是 $z$}。所以我们只要求 $x\rightarrow z$ 的上升子序列个数。枚举 $z$,发现有用的 $x$ 的总和是 $O(n)$ 的。

感觉主要差在了没有看出这个子序列没有贡献只能是恰好以 $z$ 结尾。

\subsection{「JOISC 2020 Day3」星座 3}

\subsubsection{法一}

dp:笛卡尔树上的一个区间 $[l,r]$,区间上空最多有一个星星,枚举这个星星的横坐标 $x\in[l,r]$,求出 $g_x$ 表示如果 $x$ 上空保留了一个星星,那么区间 $[l,r]$ 的最小开销,以及 $f[l,r]$ 表示上空没有星星时的最小的代价。

因为已经保留了 $x$,那么 $x_{star} \in [l,r]$ 且 $y_{star} > \max[l, r]$ 的星星肯定都不能要了,直接从 $x$ 对应的那个儿子的 $g_x$ 加上其他儿子 $f_x$ 转移来。启发式合并即可。

\subsubsection{法二}

直接考虑 dp $f[l,r]$。\textbf{建出笛卡尔树的具体形态,选择了一个 $x$ 相当于 $[l,r]$ 到 $[x,x]$ 这一条链上都不能选星星},但是这个链上的点的儿子的子树可以选,可以用树剖维护链求和。

\subsubsection{法三}

按照 $y_{star}$ 从\textbf{小到大}去考虑每一个星星,设 $pl_{star}$ 表示 $star$ 左侧第一个高于它的高楼的坐标,$pr$ 同理。那么,你要选择这颗星星,相当于要求这之后(\textbf{因为从小到大,所以只要考虑纵坐标上方的}),$[pl,pr]$ 之内都不能再有任何一个星星。

如果选择了一个星星 $x$,那么它下方所有 $x\in[pl_{star},pr_{star}]$ 的 $star$ 都必须花费 $cost_{star}$,记 $f_x$ 表示这个选择 $x$ 带来的花销。比较 $cost_x$ 与 $f_x$。

若 $f_x<cost_x$,那么 $\mathrm{Ans}\leftarrow\mathrm{Ans}+f_x$,并且令 $\forall i\in[pl_x,pr_x],f_i\leftarrow f_i+cost_x-f_x$。

否则,$\mathrm{Ans}\leftarrow\mathrm{Ans}+cost_x$。本质上其实是一个带悔贪心。

\subsection{「JOISC 2020 Day2」遗迹}

先考虑给定 $h_{1\sim 2n}$,怎么判断保留下来的是哪些元素。

\begin{theorem}[判定方式一]
    维护一个集合 $S$,对于 $i=n\sim 1$,先向 $S$ 中插入 $h_x=i$ 和 $h_y=i$ 的 $x,y$,然后取出 $S$ 中的最大值 $mx$,那么 $mx$ 最后会被保留下来,并且最终的高度为 $i$。
\end{theorem}

但是这个判定方式非常不利于计数,这时就\textbf{显式地使用需要许多日本计数题的精髓:转化判定方式,使得可以计数}。这题中,我们尝试从右向左去考虑。

\begin{theorem}[判定方式二]
    维护一个 $vis_x$ 数组,表示最终高度为 $x$ 的位置是否已经被确定。对于 $i=2n\sim 1$,判断 $vis_{h_i}$,如果已经被占用,那么令 $h_i\leftarrow h_i-1$,继续判断;若 $h_i=0$,那么表示第 $i$ 个柱子最后被夷平了。 
\end{theorem}

这个判定方式就好多了,我们考虑它的 dp 状态需要维护什么:

\begin{itemize}
    \item[1.] $vis_x$ 数组:注意到我们只要确定它的最长的等于 $1$ 的前缀的长度 $j$,剩下的可以\textbf{延迟确定}。
    \item[2.] 哪些数使用了 $0,1,2$ 次:注意到,根据 $1\sim i$ 中被夷平的柱子的数量,可以唯一确定长度为 $j$ 的 $vis_x=1$ 的前缀里,有多少个 $x$ 使用了 $2$ 次,剩下多少个用了一次;对于延迟确定的部分,可以接着延迟确定。
\end{itemize}

转移时,枚举这个前缀会延长到多长,系数可以推导和预处理。

\subsection{[AGC062D] Walk Around Neighborhood}

首先曼哈顿距离转为切比雪夫距离,会容易思考非常多。

从小到大枚举一个 $d$,判断能否全部在 $\left[-2d,2d\right]^2$ 这个矩形中;根据 $d$ 的最小性,所以肯定有经过其边界,\textbf{考虑 meet in the middle,把全部东西分成两个集合,它们分别能到达边界上}。

\begin{theorem}
    若能到达矩形 $[-2r,2r]^2$ 的边界上,则一定可以到达边上的任意一点。
\end{theorem}

这保证了 meet in the middle 时不需要考虑它具体在哪个点会和,并且使我们集中精力于讨论哪些 $r$ 是合法的。

\begin{theorem}[$r$ 合法的充要条件]
    $r$ 合法当且仅当 $\sum [d_i\le r] d_i\ge r$ 或者 $\sum [d_i\le r] d_i\ge \min\{d_i|d_i>r\}-r$。 
\end{theorem}

那么肯定是把大于 $r$ 的最小的两个数分别分进两个集合里,从小到大枚举 $r$,已经小于 $r$ 的元素的部分和用 bitset 优化背包去快速维护。

\subsection{P6072 『MdOI R1』Path}

首先,设 $dep_u$ 表示 $u$ 到根路径的异或和,我们实际上就是要求 $in_u/out_u$ 表示子树内/外的 $dep_u\oplus dep_v$ 的最大值。$in_u$ 可以用启发式合并完成,$out_u$ 发现不太能启发式合并,怎么办?

我们找到全局最大的 $dep_u\oplus dep_v$,那么只有两条链上的最大值不等于它们两个的异或,那么直接对两条链做就行了。

\subsection{海亮 Day1.B 计数}

\subsubsection{题意}

对于一个合法括号串,定义一个右括号的权值为其左侧(不包括自身)的左括号个数减去右括号个数,一个合法括号串的权值为所有右括号权值的积,求所有合法括号串权值和。

\subsubsection{题解}

这个先下降,再乘上下降前的值,这像极了求导。首先写出答案的生成函数形式:

\begin{align*}
    F_i=xF_{i-1}+F'_{i-1}
\end{align*}

设 $G_i=F_i\cdot e^{\frac{x^2}2}$,那么:

\begin{align*}
    G_i&=\left(xF_{i-1}+F'_{i-1}\right)e^{\frac{x^2}2}\\
    &=G'_{i-1}
\end{align*}

我们要求的就是 $G_0^{(2n)}[x^0]$,而 $G^{(n)}[x^i]=G^{(n-1)}[x^{i+1}](i+1)$,归纳可以得到答案。

\subsection{海亮 Day1.C 构造}

\subsubsection{题意}

将 $2,3,\cdots,3n+1$ 划分成 $n$ 个钝角三角形。

\subsubsection{题解}

首先这个钝角三角形的\textbf{条件很弱,我们找一个更严格但更简单的条件:发现 $x,y,x+y-1$ 构成钝角三角形}。

那么我们用 $(2,x,x+1),(4,x-1,x+2),(6,x-2,x+3),\cdots$ 可以搞定所有偶数;奇数是类似的。

\subsection{CCO '23 P3 - Line Town}

\begin{theorem}
    设第 $i$ 个数最后位于 $p_i$,那么它最后的值就是 $h_i\cdot (-1)^{|i-p_i|}$。
\end{theorem}

所以我们\textbf{转而考虑这个排列} $p$,最小化 $\operatorname{Inv}(p)$。

考虑按照\textbf{绝对值从大到小考虑每一种绝对值},设绝对值最大为 $x$,取出所有绝对值等于 $x$ 的位置,那么有三种数:$+x,-x,\neq x$,那么我们不妨把 $\neq x$ 的数都视为 $0$,目标变为将序列转化为 $-x,-x,\cdots,-x,0,\cdots,0,+x,\cdots,+x$,然后递归进值域缩小的问题里。

易见:这个递归唯一有后效性的变量是:\textbf{已经确定的前缀长度的奇偶性}。所以我们只需要记下这个奇偶性,然后就完全\textbf{规约为了只有 $-1,0,1$ 的问题}。

首先考虑怎么把 $0$ 去掉:注意到一个 $1$,它如果最后是 $-1$,那么代价为它左侧的 $0$ 的个数;如果最后是 $1$,那么代价为它右侧的 $0$ 的个数,其正确性来源于它要么最后是极左的要么是极右的,那么它一定会和这些 $0$ 指向的 $p$ 构成逆序对。

观察到:\textbf{非 $0$ 位置可以分为两种},第一类数是“等于 $-1$ 当且仅当它最后位于奇数位置”,第二类数是“等于 $-1$ 当且仅当它最后位于偶数位置”。\textbf{将这两种数提出来,分别排序}。

我们枚举全 $-1$ 的前缀长度(间接枚举了全 $1$ 后缀的长度),对于前缀里的奇数位置,我们要取第一类数的一个前缀;对于前缀里的偶数位置,我们要取第二类数的一个前缀;对于后缀里的奇数位置,我们要取第二类数的一个后缀;对于后缀里的偶数位置,我们要取第一类数的一个后缀。我们在枚举长度的时候,动态维护它能不能刚好用完所有非 $0$ 位置,以及当前的排列的逆序对数。

\subsection{IOI 2023 国家队集训@威海 Day 2 野餐}

\subsubsection{题意}

有一个循环排列 $p_{0\sim n-1}$,进行如下三轮游戏:

\begin{itemize}
    \item $n$ 个策略相同的骑士,分别坐在 $p_{0\sim n-1}$ 前,坐在 $p_i$ 前的骑士只能知道 $p_i,p_{(i+1)\bmod n}$ 的值;每个骑士要写下一个值 $b_i\in[1,10^9]$。
    \item $n$ 个策略相同的骑士,分别坐在 $b_{0\sim n-1}$ 前,坐在 $p_i$ 前的骑士只能知道 $b_i,b_{(i+1)\bmod n},b_{(i-1)\bmod n}$ 的值;每个骑士要写下一个值 $c_i\in[0,10^9)$。
    \item $n$ 个策略相同的骑士,分别坐在 $c_{0\sim n-1}$ 前,坐在 $p_i$ 前的骑士只能知道 $c_i,c_{(i+1)\bmod n},c_{(i-1)\bmod n}$ 的值;每个骑士要写下一个值 $d_i\in[0,3)$。
\end{itemize}

注意,每个骑士不知道 $n$ 的具体值,你需要给每个骑士设计一个固定的策略,使得 $\forall i,d_i\neq d_{(i+1)\bmod n}$。$n\le 10^5$。

\subsubsection{题解}

由于不知道 $n$ 的具体值,所以\textbf{一个骑士知道的信息非常地片面,所以我们不妨以一个非常局部的性质作为条件和目标,逐渐递归}。考虑完成这样一件事情:$p,b,c,d$ 都需要满足 $x_i\neq x_{(i+1)\bmod}$,要在满足此条件的前提下逐渐缩小其值域。

考虑我们求出一个函数 $u\neq v,v\neq w \Rightarrow f(u,v)\neq f(v,w)$。

\begin{definition}[$f$ 的构造一]
    \textbf{设 $u,v$ 二进制下第一个不相同的位为 $2^{i}$,$f(u,v)=2i+u_i$}。这个 $f$ 能做到从 $2^T$ 映射到 $2T$。
\end{definition}

\begin{definition}[$f$ 的构造二]
    可以先把 $u,v$ 映射至一个 $T$ 个 $0$ 与 $T$ 个 $1$ 的整数,然后再求上面的 $f$。这个 $f$ 能做到从 $\binom{2T}{T}$ 映射到 $2T$。
\end{definition}

令 $B(x,r)=f(x,r,10)$,$C(l,x,r)=f(f(l,x,3),f(l,x,3),2)$ 就可以在两轮内把值域缩小到 $[0,3]$。考虑最后一轮 $D(l,x,r)$,若 $x\le 2$,$D(x,l,r)=x$;否则,$l,r\le 2$,于是 $l,r$ 两个数一定不会变,那么 $x$ 就是 $[0,2]$ 中不与 $l,r$ 相撞的一个数。

\subsection{IOI 2023 国家队集训@威海 Day 2 外环路 2}

\subsubsection{题意}

有一棵树,保证点的标号为一个以 $1$ 开始的 dfs 序,将所有叶子节点按照编号排序,前后两两相接串成环,形成了一个外环内树的平面图。你要截断尽可能少的边,使得这个图不存在简单奇环。$N\le 10^5$。

\subsubsection{题解}

注意到一个奇环,一定是一个非树边加上偶数个树边,也就是说,如果将树上所有点黑白染色,非树边的两个端点的颜色一定不能相同。

转化问题为:给每个点染上一个颜色,最小化两端点颜色相同的边的个数。树形 dp 即可。

\end{document}
